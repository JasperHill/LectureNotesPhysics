\title{In-medium SRG approaches to infinite nuclear matter}
\author{Scott K.~Bogner, Heiko Hergert, Titus Morris, Nathan Parzuchowski, and Fei Yuan}
\institute{Scott Bogner  \at Department of Physics and Astronomy and National Superconducting Cyclotron Laboratory, Michigan State University, East Lansing, Michigan USA, \email{bogner@nscl.msu.edu}, \and Heiko Hergert  \at Department of Physics and Astronomy and National Superconducting Cyclotron Laboratory, Michigan State University, East Lansing, Michigan USA, \email{hergert@nscl.msu.edu},\and Titus Morris  \at Department of Physics and Astronomy and National Superconducting Cyclotron Laboratory, Michigan State University, East Lansing, Michigan USA, \email{morrist@nscl.msu.edu},\and Nathan Parzuchowski  \at Department of Physics and Astronomy and National Superconducting Cyclotron Laboratory, Michigan State University, East Lansing, Michigan USA, \email{parzuchowski@frib.msu.edu}, \and Fei Yuan  \at Department of Physics and Astronomy and National Superconducting Cyclotron Laboratory, Michigan State University, East Lansing, Michigan USA, \email{parzuchowski@frib.msu.edu}}
\maketitle
\abstract{Each chapter should be preceded by an abstract (10--15 lines long) that summarizes the content. The abstract will appear \textit{online} at \url{www.SpringerLink.com} and be available with unrestricted access. This allows unregistered users to read the abstract as a teaser for the complete chapter. As a general rule the abstracts will not appear in the printed version of your book unless it is the style of your particular book or that of the series to which your book belongs.\newline\indent
Please use the 'starred' version of the new Springer \texttt{abstract} command for typesetting the text of the online abstracts (cf. source file of this chapter template \texttt{abstract}) and include them with the source files of your manuscript. Use the plain \texttt{abstract} command if the abstract is also to appear in the printed version of the book.}

%\maketile


\section{Introduction}
\section{The similarity renormalization group approach}

\section{In-medium SRG studies of infinite matter}

\begin{acknowledgement}
If you want to include acknowledgments of assistance and the like at the end of an individual chapter please use the \verb|acknowledgement| environment -- it will automatically render Springer's preferred layout.
\end{acknowledgement}
%







 
