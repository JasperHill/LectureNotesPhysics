\chapter{Wave function based many-body methods}
\author{Morten Hjorth-Jensen}
\institute{Morten Hjorth-Jensen \at Name of institution and address, \email{name@email.address}}

\abstract{Here we present and discuss various many-body methods } 


%\maketile

\section{Introduction}
\section{Hartree-Fock theory}
\section{Full configuration interaction theory}
\subsection{Slater determinants as basis states}

The simplest possible choice for many-body wavefunctions are \textbf{product} wavefunctions.
That is
\[ 
\Psi(x_1, x_2, x_3, \ldots, x_A) \approx \phi_1(x_1) \phi_2(x_2) \phi_3(x_3) \ldots
\]
because we are really only good  at thinking about one particle at a time. Such 
product wavefunctions, without correlations, are easy to 
work with; for example, if the single-particle states $\phi_i(x)$ are orthonormal, then 
the product wavefunctions are easy to orthonormalize.   

Similarly, computing matrix elements of operators are relatively easy, because the 
integrals factorize.


The price we pay is the lack of correlations, which we must build up by using many, many product 
wavefunctions.


Because we have fermions, we are required to have antisymmetric wavefunctions, that is
\[
\Psi(x_1, x_2, x_3, \ldots, x_A) = - \Psi(x_2, x_1, x_3, \ldots, x_A)
\]
etc. This is accomplished formally by using the determinantal formalism
\[
\Psi(x_1, x_2, \ldots, x_A) 
= \frac{1}{\sqrt{A!}} 
\det \left | 
\begin{array}{cccc}
\phi_1(x_1) & \phi_1(x_2) & \ldots & \phi_1(x_A) \\
\phi_2(x_1) & \phi_2(x_2) & \ldots & \phi_2(x_A) \\
 \vdots & & &  \\
\phi_A(x_1) & \phi_A(x_2) & \ldots & \phi_A(x_A) 
\end{array}
\right |
\]
Product wavefunction + antisymmetry (Pauli principle) = Slater determinant. 


Properties of the determinant (interchange of any two rows or 
any two columns yields a change in sign; thus no two rows and no 
two columns can be the same) lead to the following consequence of the Pauli principle:

\begin{itemize}
\item No two particles can be at the same place (two columns the same); and

\item No two particles can be in the same state (two rows the same).
\end{itemize}

\noindent
As a practical matter, however, Slater determinants beyond $N=4$ quickly become 
unwieldy. Thus we turn to the \textbf{occupation representation} or \textbf{second quantization} to simplify calculations. 

The occupation representation, using fermion \textbf{creation} and \textbf{annihilation} 
operators, is compact and efficient. It is also abstract and, at first encounter, not easy to 
internalize. It is inspired by other operator formalism, such as the ladder operators for 
the harmonic oscillator or for angular momentum, but unlike those cases, the operators \textbf{do not have coordinate space representations}.

Instead, one can think of fermion creation/annihilation operators as a game of symbols that 
compactly reproduces what one would do, albeit clumsily, with full coordinate-space Slater 
determinants. 



We start with a set of orthonormal single-particle states $\{ \phi_i(x) \}$. 
(Note: this requirement, and others, can be relaxed, but leads to a 
more involved formalism.) \textbf{Any} orthonormal set will do. 

To each single-particle state $\phi_i(x)$ we associate a creation operator 
$\hat{a}^\dagger_i$ and an annihilation operator $\hat{a}_i$. 

When acting on the vacuum state $| 0 \rangle$, the creation operator $\hat{a}^\dagger_i$ causes 
a particle to occupy the single-particle state $\phi_i(x)$:
\[
\phi_i(x) \rightarrow \hat{a}^\dagger_i |0 \rangle
\]



But with multiple creation operators we can occupy multiple states:
\[
\phi_i(x) \phi_j(x^\prime) \phi_k(x^{\prime \prime}) 
\rightarrow \hat{a}^\dagger_i \hat{a}^\dagger_j \hat{a}^\dagger_k |0 \rangle.
\]

Now we impose antisymmetry, by having the fermion operators satisfy  \textbf{anticommutation relations}:
\[
\hat{a}^\dagger_i \hat{a}^\dagger_j + \hat{a}^\dagger_j \hat{a}^\dagger_i
= [ \hat{a}^\dagger_i ,\hat{a}^\dagger_j ]_+ 
= \{ \hat{a}^\dagger_i ,\hat{a}^\dagger_j \} = 0
\]
so that 
\[
\hat{a}^\dagger_i \hat{a}^\dagger_j = - \hat{a}^\dagger_j \hat{a}^\dagger_i
\]




Because of this property, automatically $\hat{a}^\dagger_i \hat{a}^\dagger_i = 0$, 
enforcing the Pauli exclusion principle.  Thus when writing a Slater determinant 
using creation operators, 
\[
\hat{a}^\dagger_i \hat{a}^\dagger_j \hat{a}^\dagger_k \ldots |0 \rangle
\]
each index $i,j,k, \ldots$ must be unique.




\subsection{Full Configuration Interaction Theory}

We have defined the ansatz for the ground state as 
\[
|\Phi_0\rangle = \left(\prod_{i\le F}\hat{a}_{i}^{\dagger}\right)|0\rangle,
\]
where the index $i$ defines different single-particle states up to the Fermi level. We have assumed that we have $N$ fermions. 
A given one-particle-one-hole ($1p1h$) state can be written as
\[
|\Phi_i^a\rangle = \hat{a}_{a}^{\dagger}\hat{a}_i|\Phi_0\rangle,
\]
while a $2p2h$ state can be written as
\[
|\Phi_{ij}^{ab}\rangle = \hat{a}_{a}^{\dagger}\hat{a}_{b}^{\dagger}\hat{a}_j\hat{a}_i|\Phi_0\rangle,
\]
and a general $ApAh$ state as 
\[
|\Phi_{ijk\dots}^{abc\dots}\rangle = \hat{a}_{a}^{\dagger}\hat{a}_{b}^{\dagger}\hat{a}_{c}^{\dagger}\dots\hat{a}_k\hat{a}_j\hat{a}_i|\Phi_0\rangle.
\]

We use letters $ijkl\dots$ for states below the Fermi level and $abcd\dots$ for states above the Fermi level. A general single-particle state is given by letters $pqrs\dots$.

We can then expand our exact state function for the ground state 
as
\[
|\Psi_0\rangle=C_0|\Phi_0\rangle+\sum_{ai}C_i^a|\Phi_i^a\rangle+\sum_{abij}C_{ij}^{ab}|\Phi_{ij}^{ab}\rangle+\dots
=(C_0+\hat{C})|\Phi_0\rangle,
\]
where we have introduced the so-called correlation operator 
\[
\hat{C}=\sum_{ai}C_i^a\hat{a}_{a}^{\dagger}\hat{a}_i  +\sum_{abij}C_{ij}^{ab}\hat{a}_{a}^{\dagger}\hat{a}_{b}^{\dagger}\hat{a}_j\hat{a}_i+\dots
\]
Since the normalization of $\Psi_0$ is at our disposal and since $C_0$ is by hypothesis non-zero, we may arbitrarily set $C_0=1$ with 
corresponding proportional changes in all other coefficients. Using this so-called intermediate normalization we have
\[
\langle \Psi_0 | \Phi_0 \rangle = \langle \Phi_0 | \Phi_0 \rangle = 1, 
\]
resulting in 
\[
|\Psi_0\rangle=(1+\hat{C})|\Phi_0\rangle.
\]


We rewrite 
\[
|\Psi_0\rangle=C_0|\Phi_0\rangle+\sum_{ai}C_i^a|\Phi_i^a\rangle+\sum_{abij}C_{ij}^{ab}|\Phi_{ij}^{ab}\rangle+\dots,
\]
in a more compact form as 
\[
|\Psi_0\rangle=\sum_{PH}C_H^P\Phi_H^P=\left(\sum_{PH}C_H^P\hat{A}_H^P\right)|\Phi_0\rangle,
\]
where $H$ stands for $0,1,\dots,n$ hole states and $P$ for $0,1,\dots,n$ particle states. 
Our requirement of unit normalization gives
\[
\langle \Psi_0 | \Psi_0 \rangle = \sum_{PH}|C_H^P|^2= 1,
\]
and the energy can be written as 
\[
E= \langle \Psi_0 | \hat{H} |\Psi_0 \rangle= \sum_{PP'HH'}C_H^{*P}\langle \Phi_H^P | \hat{H} |\Phi_{H'}^{P'} \rangle C_{H'}^{P'}.
\]


Normally 
\[
E= \langle \Psi_0 | \hat{H} |\Psi_0 \rangle= \sum_{PP'HH'}C_H^{*P}\langle \Phi_H^P | \hat{H} |\Phi_{H'}^{P'} \rangle C_{H'}^{P'},
\]
is solved by diagonalization setting up the Hamiltonian matrix defined by the basis of all possible Slater determinants. A diagonalization
is equivalent to finding the variational minimum   of 
\[
 \langle \Psi_0 | \hat{H} |\Psi_0 \rangle-\lambda \langle \Psi_0 |\Psi_0 \rangle,
\]
where $\lambda$ is a variational multiplier to be identified with the energy of the system.

The minimization process results in 
\[
\delta\left[ \langle \Psi_0 | \hat{H} |\Psi_0 \rangle-\lambda \langle \Psi_0 |\Psi_0 \rangle\right]=
\]
\[
\sum_{P'H'}\left\{\delta[C_H^{*P}]\langle \Phi_H^P | \hat{H} |\Phi_{H'}^{P'} \rangle C_{H'}^{P'}+
C_H^{*P}\langle \Phi_H^P | \hat{H} |\Phi_{H'}^{P'} \rangle \delta[C_{H'}^{P'}]-
\lambda( \delta[C_H^{*P}]C_{H'}^{P'}+C_H^{*P}\delta[C_{H'}^{P'}]\right\} = 0.
\]
Since the coefficients $\delta[C_H^{*P}]$ and $\delta[C_{H'}^{P'}]$ are complex conjugates it is necessary and sufficient to require the quantities that multiply with $\delta[C_H^{*P}]$ to vanish.  

This leads to 
\[
\sum_{P'H'}\langle \Phi_H^P | \hat{H} |\Phi_{H'}^{P'} \rangle C_{H'}^{P'}-\lambda C_H^{P}=0,
\]
for all sets of $P$ and $H$.

If we then multiply by the corresponding $C_H^{*P}$ and sum over $PH$ we obtain
\[ 
\sum_{PP'HH'}C_H^{*P}\langle \Phi_H^P | \hat{H} |\Phi_{H'}^{P'} \rangle C_{H'}^{P'}-\lambda\sum_{PH}|C_H^P|^2=0,
\]
leading to the identification $\lambda = E$. This means that we have for all $PH$ sets
\begin{equation}
\sum_{P'H'}\langle \Phi_H^P | \hat{H} -E|\Phi_{H'}^{P'} \rangle = 0. \label{eq:fullci}
\end{equation}



An alternative way to derive the last equation is to start from 
\[
(\hat{H} -E)|\Psi_0\rangle = (\hat{H} -E)\sum_{P'H'}C_{H'}^{P'}|\Phi_{H'}^{P'} \rangle=0, 
\]
and if this equation is successively projected against all $\Phi_H^P$ in the expansion of $\Psi$, we end up with Eq.~(\ref{eq:fullci}).

One solves this equation normally by diagonalization. If we are able to solve this equation exactly (that is
numerically exactly) in a large Hilbert space (it will be truncated in terms of the number of single-particle states included in the definition
of Slater determinants), it can then serve as a benchmark for other many-body methods which approximate the correlation operator
$\hat{C}$.  


\subsection{Example of a Hamiltonian matrix}

Suppose, as an example, that we have six fermions below the Fermi level.
This means that we can make at most $6p-6h$ excitations. If we have an infinity of single particle states above the Fermi level, we will obviously have an infinity of say $2p-2h$ excitations. Each such way to configure the particles is called a \textbf{configuration}. We will always have to truncate in the basis of single-particle states.
This gives us a finite number of possible Slater determinants. Our Hamiltonian matrix would then look like (where each block can have a large dimensionalities):


\begin{quote}
\begin{tabular}{cccccccc}
\hline
\multicolumn{1}{c}{  } & \multicolumn{1}{c}{ $0p-0h$ } & \multicolumn{1}{c}{ $1p-1h$ } & \multicolumn{1}{c}{ $2p-2h$ } & \multicolumn{1}{c}{ $3p-3h$ } & \multicolumn{1}{c}{ $4p-4h$ } & \multicolumn{1}{c}{ $5p-5h$ } & \multicolumn{1}{c}{ $6p-6h$ } \\
\hline
$0p-0h$ & x       & x       & x       & 0       & 0       & 0       & 0       \\
$1p-1h$ & x       & x       & x       & x       & 0       & 0       & 0       \\
$2p-2h$ & x       & x       & x       & x       & x       & 0       & 0       \\
$3p-3h$ & 0       & x       & x       & x       & x       & x       & 0       \\
$4p-4h$ & 0       & 0       & x       & x       & x       & x       & x       \\
$5p-5h$ & 0       & 0       & 0       & x       & x       & x       & x       \\
$6p-6h$ & 0       & 0       & 0       & 0       & x       & x       & x       \\
\hline
\end{tabular}
\end{quote}

\noindent
with a two-body force. Why are there non-zero blocks of elements? 
If we use a Hartree-Fock basis, this corresponds to a particular unitary transformation where matrix elements of the type $\langle 0p-0h \vert \hat{H} \vert 1p-1h\rangle =\langle \Phi_0 | \hat{H}|\Phi_{i}^{a}\rangle=0$ and our Hamiltonian matrix becomes 


\begin{quote}
\begin{tabular}{cccccccc}
\hline
\multicolumn{1}{c}{  } & \multicolumn{1}{c}{ $0p-0h$ } & \multicolumn{1}{c}{ $1p-1h$ } & \multicolumn{1}{c}{ $2p-2h$ } & \multicolumn{1}{c}{ $3p-3h$ } & \multicolumn{1}{c}{ $4p-4h$ } & \multicolumn{1}{c}{ $5p-5h$ } & \multicolumn{1}{c}{ $6p-6h$ } \\
\hline
$0p-0h$ & $\tilde{x}$ & 0           & $\tilde{x}$ & 0           & 0           & 0           & 0           \\
$1p-1h$ & 0           & $\tilde{x}$ & $\tilde{x}$ & $\tilde{x}$ & 0           & 0           & 0           \\
$2p-2h$ & $\tilde{x}$ & $\tilde{x}$ & $\tilde{x}$ & $\tilde{x}$ & $\tilde{x}$ & 0           & 0           \\
$3p-3h$ & 0           & $\tilde{x}$ & $\tilde{x}$ & $\tilde{x}$ & $\tilde{x}$ & $\tilde{x}$ & 0           \\
$4p-4h$ & 0           & 0           & $\tilde{x}$ & $\tilde{x}$ & $\tilde{x}$ & $\tilde{x}$ & $\tilde{x}$ \\
$5p-5h$ & 0           & 0           & 0           & $\tilde{x}$ & $\tilde{x}$ & $\tilde{x}$ & $\tilde{x}$ \\
$6p-6h$ & 0           & 0           & 0           & 0           & $\tilde{x}$ & $\tilde{x}$ & $\tilde{x}$ \\
\hline
\end{tabular}
\end{quote}

\noindent
If we do not make any truncations in the possible sets of Slater determinants (many-body states) we can make by distributing $A$ nucleons among $n$ single-particle states, we call such a calculation for 
\begin{itemize}
\item Full configuration interaction theory
\end{itemize}

\noindent
If we make truncations, we have different possibilities

\begin{itemize}
\item The standard nuclear shell-model. Here we define an effective Hilbert space with respect to a given core. The calculations are normally then performed for all many-body states that can be constructed from the effective Hilbert spaces. This approach requires a properly defined effective Hamiltonian

\item We can truncate in the number of excitations. For example, we can limit the possible Slater determinants to only $1p-1h$ and $2p-2h$ excitations. This is called a configuration interaction calculation at the level of singles and doubles excitations, or just CISD. 

\item We can limit the number of excitations in terms of the excitation energies. If we do not define a core, this defines normally what is called the no-core shell-model approach. 
\end{itemize}

\noindent
What happens if we have a three-body interaction and a Hartree-Fock basis? 

Full configuration interaction theory calculations provide in principle, if we can diagonalize numerically, all states of interest. The dimensionality of the problem explodes however quickly.

The total number of Slater determinants which can be built with say $N$ neutrons distributed among $n$ single particle states is
\[
\left (\begin{array}{c} n \\ N\end{array} \right) =\frac{n!}{(n-N)!N!}. 
\]

For a model space which comprises the first for major shells only $0s$, $0p$, $1s0d$ and $1p0f$ we have $40$ single particle states for neutrons and protons.  For the eight neutrons of oxygen-16 we would then have
\[
\left (\begin{array}{c} 40 \\ 8\end{array} \right) =\frac{40!}{(32)!8!}\sim 10^{9}, 
\]
and multiplying this with the number of proton Slater determinants we end up with approximately witha dimensionality $d$ of $d\sim 10^{18}$.


This number can be reduced if we look at specific symmetries only. However, the dimensionality explodes quickly!

\begin{itemize}
\item For Hamiltonian matrices of dimensionalities  which are smaller than $d\sim 10^5$, we would use so-called direct methods for diagonalizing the Hamiltonian matrix

\item For larger dimensionalities iterative eigenvalue solvers like Lanczos' method are used. The most efficient codes at present can handle matrices of $d\sim 10^{10}$. 
\end{itemize}

\noindent
\subsection{A non-practical way of solving the eigenvalue problem}

For reasons to come (links with Coupled-Cluster theory and Many-Body perturbation theory), 
we will rewrite Eq.~(\ref{eq:fullci}) as a set of coupled non-linear equations in terms of the unknown coefficients $C_H^P$. 
To obtain the eigenstates and eigenvalues in terms of non-linear equations is not a very practical approach. However, it serves the scope of linking FCI theory with approximative solutions to the many-body problem.

To see this, we look at the contributions arising from 
\[
\langle \Phi_H^P | = \langle \Phi_0|
\]
in  Eq.~(\ref{eq:fullci}), that is we multiply with $\langle \Phi_0 |$
from the left in 
\[
(\hat{H} -E)\sum_{P'H'}C_{H'}^{P'}|\Phi_{H'}^{P'} \rangle=0. 
\]
If we assume that we have a two-body operator at most, Slater's rule gives then an equation for the 
correlation energy in terms of $C_i^a$ and $C_{ij}^{ab}$ only.  We get then
\[
\langle \Phi_0 | \hat{H} -E| \Phi_0\rangle + \sum_{ai}\langle \Phi_0 | \hat{H} -E|\Phi_{i}^{a} \rangle C_{i}^{a}+
\sum_{abij}\langle \Phi_0 | \hat{H} -E|\Phi_{ij}^{ab} \rangle C_{ij}^{ab}=0,
\]
or 
\[
E-E_0 =\Delta E=\sum_{ai}\langle \Phi_0 | \hat{H}|\Phi_{i}^{a} \rangle C_{i}^{a}+
\sum_{abij}\langle \Phi_0 | \hat{H}|\Phi_{ij}^{ab} \rangle C_{ij}^{ab},
\]
where the energy $E_0$ is the reference energy and $\Delta E$ defines the so-called correlation energy.
The single-particle basis functions  could be the results of a Hartree-Fock calculation or just the eigenstates of the non-interacting part of the Hamiltonian. 

In our notes on Hartree-Fock calculations, 
we have already computed the matrix $\langle \Phi_0 | \hat{H}|\Phi_{i}^{a}\rangle $ and $\langle \Phi_0 | \hat{H}|\Phi_{ij}^{ab}\rangle$.  If we are using a Hartree-Fock basis, then the matrix elements
$\langle \Phi_0 | \hat{H}|\Phi_{i}^{a}\rangle=0$ and we are left with a \emph{correlation energy} given by
\[
E-E_0 =\Delta E^{HF}=\sum_{abij}\langle \Phi_0 | \hat{H}|\Phi_{ij}^{ab} \rangle C_{ij}^{ab}. 
\]


Inserting the various matrix elements we can rewrite the previous equation as
\[
\Delta E=\sum_{ai}\langle i| \hat{f}|a \rangle C_{i}^{a}+
\sum_{abij}\langle ij | \hat{v}| ab \rangle C_{ij}^{ab}.
\]
This equation determines the correlation energy but not the coefficients $C$. 
We need more equations. Our next step is to set up
\[
\langle \Phi_i^a | \hat{H} -E| \Phi_0\rangle + \sum_{bj}\langle \Phi_i^a | \hat{H} -E|\Phi_{j}^{b} \rangle C_{j}^{b}+
\sum_{bcjk}\langle \Phi_i^a | \hat{H} -E|\Phi_{jk}^{bc} \rangle C_{jk}^{bc}+
\sum_{bcdjkl}\langle \Phi_i^a | \hat{H} -E|\Phi_{jkl}^{bcd} \rangle C_{jkl}^{bcd}=0,
\]
as this equation will allow us to find an expression for the coefficents $C_i^a$ since we can rewrite this equation as 
\[
\langle i | \hat{f}| a\rangle +\langle \Phi_i^a | \hat{H}|\Phi_{i}^{a} \rangle C_{i}^{a}+ \sum_{bj\ne ai}\langle \Phi_i^a | \hat{H}|\Phi_{j}^{b} \rangle C_{j}^{b}+
\sum_{bcjk}\langle \Phi_i^a | \hat{H}|\Phi_{jk}^{bc} \rangle C_{jk}^{bc}+
\sum_{bcdjkl}\langle \Phi_i^a | \hat{H}|\Phi_{jkl}^{bcd} \rangle C_{jkl}^{bcd}=EC_i^a.
\]

We see that on the right-hand side we have the energy $E$. This leads to a non-linear equation in the unknown coefficients. 
These equations are normally solved iteratively ( that is we can start with a guess for the coefficients $C_i^a$). A common choice is to use perturbation theory for the first guess, setting thereby
\[
 C_{i}^{a}=\frac{\langle i | \hat{f}| a\rangle}{\epsilon_i-\epsilon_a}.
\]

The observant reader will however see that we need an equation for $C_{jk}^{bc}$ and $C_{jkl}^{bcd}$ as well.
To find equations for these coefficients we need then to continue our multiplications from the left with the various
$\Phi_{H}^P$ terms. 


For $C_{jk}^{bc}$ we need then
\[
\langle \Phi_{ij}^{ab} | \hat{H} -E| \Phi_0\rangle + \sum_{kc}\langle \Phi_{ij}^{ab} | \hat{H} -E|\Phi_{k}^{c} \rangle C_{k}^{c}+
\]
\[
\sum_{cdkl}\langle \Phi_{ij}^{ab} | \hat{H} -E|\Phi_{kl}^{cd} \rangle C_{kl}^{cd}+\sum_{cdeklm}\langle \Phi_{ij}^{ab} | \hat{H} -E|\Phi_{klm}^{cde} \rangle C_{klm}^{cde}+\sum_{cdefklmn}\langle \Phi_{ij}^{ab} | \hat{H} -E|\Phi_{klmn}^{cdef} \rangle C_{klmn}^{cdef}=0,
\]
and we can isolate the coefficients $C_{kl}^{cd}$ in a similar way as we did for the coefficients $C_{i}^{a}$. 
A standard choice for the first iteration is to set 
\[
C_{ij}^{ab} =\frac{\langle ij \vert \hat{v} \vert ab \rangle}{\epsilon_i+\epsilon_j-\epsilon_a-\epsilon_b}.
\]
At the end we can rewrite our solution of the Schroedinger equation in terms of $n$ coupled equations for the coefficients $C_H^P$.
This is a very cumbersome way of solving the equation. However, by using this iterative scheme we can illustrate how we can compute the
various terms in the wave operator or correlation operator $\hat{C}$. We will later identify the calculation of the various terms $C_H^P$
as parts of different many-body approximations to full CI. In particular, we can  relate this non-linear scheme with Coupled Cluster theory and
many-body perturbation theory.


\subsection{Summarizing FCI and bringing in approximative methods}


If we can diagonalize large matrices, FCI is the method of choice since:
\begin{itemize}
\item It gives all eigenvalues, ground state and excited states

\item The eigenvectors are obtained directly from the coefficients $C_H^P$ which result from the diagonalization

\item We can compute easily expectation values of other operators, as well as transition probabilities

\item Correlations are easy to understand in terms of contributions to a given operator beyond the Hartree-Fock contribution. This is the standard approach in  many-body theory. 
\end{itemize}

\noindent
The correlation energy is defined as, with a two-body Hamiltonian,  
\[
\Delta E=\sum_{ai}\langle i| \hat{f}|a \rangle C_{i}^{a}+
\sum_{abij}\langle ij | \hat{v}| ab \rangle C_{ij}^{ab}.
\]

The coefficients $C$ result from the solution of the eigenvalue problem. 
The energy of say the ground state is then
\[
E=E_{ref}+\Delta E,
\]
where the so-called reference energy is the energy we obtain from a Hartree-Fock calculation, that is
\[
E_{ref}=\langle \Phi_0 \vert \hat{H} \vert \Phi_0 \rangle.
\]

However, as we have seen, even for a small case like the four first major shells and a nucleus like oxygen-16, the dimensionality becomes quickly intractable. If we wish to include single-particle states that reflect weakly bound systems, we need a much larger single-particle basis. We need thus approximative methods that sum specific correlations to infinite order. 

Popular methods are
\begin{itemize}
\item Many-body perturbation theory (in essence a Taylor expansion)

\item Coupled cluster theory (coupled non-linear equations)

\item Green's function approaches (matrix inversion)

\item Similarity group transformation methods (coupled ordinary differential equations
\end{itemize}

\noindent
All these methods start normally with a Hartree-Fock basis as the calculational basis. 


\subsection{Building a many-body basis}

Here we will discuss how we can set up a single-particle basis which we can use in the various parts of our projects, from the simple pairing model to infinite nuclear matter. We will use here the simple pairing model to illustrate in particular how to set up a single-particle basis. We will also use this do discuss standard FCI approaches like:
\begin{enumerate}
 \item Standard shell-model basis in one or two major shells

 \item Full CI in a given basis and no truncations

 \item CISD and CISDT approximations

 \item No-core shell model and truncation in excitation energy
\end{enumerate}

\noindent
An important step in an FCI code  is to construct the many-body basis.  

While the formalism is independent of the choice of basis, the \textbf{effectiveness} of a calculation 
will certainly be basis dependent. 

Furthermore there are common conventions useful to know.

First, the single-particle basis has angular momentum as a good quantum number.  You can 
imagine the single-particle wavefunctions being generated by a one-body Hamiltonian, 
for example a harmonic oscillator.  Modifications include harmonic oscillator plus 
spin-orbit splitting, or self-consistent mean-field potentials, or the Woods-Saxon potential which mocks 
up the self-consistent mean-field. 
For nuclei, the harmonic oscillator, modified by spin-orbit splitting, provides a useful language 
for describing single-particle states.


Each single-particle state is labeled by the following quantum numbers: 

\begin{itemize}
\item Orbital angular momentum $l$

\item Intrinsic spin $s$ = 1/2 for protons and neutrons

\item Angular momentum $j = l \pm 1/2$

\item $z$-component $j_z$ (or $m$)

\item Some labeling of the radial wavefunction, typically $n$ the number of nodes in  the radial wavefunction, but in the case of harmonic oscillator one can also use the principal quantum number $N$, where the harmonic oscillator energy is $(N+3/2)\hbar \omega$.  For our nuclear matter projects, you will need to change the quantum numbers to those relevant for calculations
\end{itemize}

\noindent
in three-dimensional cartesian basis, see the relevante \href{{https://github.com/NuclearTalent/Course2ManyBodyMethods/blob/master/doc/pub/cc/pdf/Lectures1-2_TALENT_NuclearMatter_GH.pdf}}{lectures}.


In this format one labels states by $n(l)_j$, with $(l)$ replaced by a letter:
$s$ for $l=0$, $p$ for $l=1$, $d$ for $l=2$, $f$ for $l=3$, and thenceforth alphabetical.


 In practice the single-particle space has to be severely truncated.  This truncation is 
typically based upon the single-particle energies, which is the effective energy 
from a mean-field potential. 

Sometimes we freeze the core and only consider a valence space. For example, one 
may assume a frozen ${}^{4}\mbox{He}$ core, with two protons and two neutrons in the $0s_{1/2}$ 
shell, and then only allow active particles in the $0p_{1/2}$ and $0p_{3/2}$ orbits. 


Another example is a frozen ${}^{16}\mbox{O}$ core, with eight protons and eight neutrons filling the 
$0s_{1/2}$,  $0p_{1/2}$ and $0p_{3/2}$ orbits, with valence particles in the 
$0d_{5/2}, 1s_{1/2}$ and $0d_{3/2}$ orbits.


Sometimes we refer to nuclei by the valence space where their last nucleons go.  
So, for example, we call ${}^{12}\mbox{C}$ a $p$-shell nucleus, while ${}^{26}\mbox{Al}$ is an 
$sd$-shell nucleus and ${}^{56}\mbox{Fe}$ is a $pf$-shell nucleus.





There are different kinds of truncations.

\begin{itemize}
\item For example, one can start with `filled' orbits (almost always the lowest), and then  allow one, two, three... particles excited out of those filled orbits. These are called  1p-1h, 2p-2h, 3p-3h excitations. 

\item Alternately, one can state a maximal orbit and allow all possible configurations with  particles occupying states up to that maximum. This is called \emph{full configuration}.

\item Finally, for particular use in nuclear physics, there is the \emph{energy} truncation, also  called the $N\hbar\Omega$ or $N_{max}$ truncation. 
\end{itemize}

\noindent
Here one works in a harmonic oscillator basis, with each major oscillator shell assigned  a principal quantum number $N=0,1,2,3,...$. 
The $N\hbar\Omega$ or $N_{max}$ truncation: Any configuration is given an noninteracting energy, which is the sum 
of the single-particle harmonic oscillator energies. (Thus this ignores 
spin-orbit splitting.)

Excited state are labeled relative to the lowest configuration by the 
number of harmonic oscillator quanta.

This truncation is useful because if one includes \emph{all} configuration up to 
some $N_{max}$, and has a translationally invariant interaction, then the intrinsic 
motion and the center-of-mass motion factor. In other words, we can know exactly 
the center-of-mass wavefunction. 

In almost all cases, the many-body Hamiltonian is rotationally invariant. This means 
it commutes with the operators $\hat{J}^2, \hat{J}_z$ and so eigenstates will have 
good $J,M$. Furthermore, the eigenenergies do not depend upon the orientation $M$. 


Therefore we can choose to construct a many-body basis which has fixed $M$; this is 
called an $M$-scheme basis. 


Alternately, one can construct a many-body basis which has fixed $J$, or a $J$-scheme 
basis. 

The Hamiltonian matrix will have smaller dimensions (a factor of 10 or more) in the $J$-scheme than in the $M$-scheme. 
On the other hand, as we'll show in the next slide, the $M$-scheme is very easy to 
construct with Slater determinants, while the $J$-scheme basis states, and thus the 
matrix elements, are more complicated, almost always being linear combinations of 
$M$-scheme states. $J$-scheme bases are important and useful, but we'll focus on the 
simpler $M$-scheme.

The quantum number $m$ is additive (because the underlying group is Abelian): 
if a Slater determinant $\hat{a}_i^\dagger \hat{a}^\dagger_j \hat{a}^\dagger_k \ldots | 0 \rangle$ 
is built from single-particle states all with good $m$, then the total 
\[
M = m_i + m_j + m_k + \ldots
\]
This is \emph{not} true of $J$, because the angular momentum group SU(2) is not Abelian.

The upshot is that 
\begin{itemize}
\item It is easy to construct a Slater determinant with good total $M$;

\item It is trivial to calculate $M$ for each Slater determinant;

\item So it is easy to construct an $M$-scheme basis with fixed total $M$.
\end{itemize}

\noindent
Note that the individual $M$-scheme basis states will \emph{not}, in general, 
have good total $J$. 
Because the Hamiltonian is rotationally invariant, however, the eigenstates will 
have good $J$. (The situation is muddied when one has states of different $J$ that are 
nonetheless degenerate.) 




Example: two $j=1/2$ orbits


\begin{quote}
\begin{tabular}{ccccc}
\hline
\multicolumn{1}{c}{ Index } & \multicolumn{1}{c}{ $n$ } & \multicolumn{1}{c}{ $l$ } & \multicolumn{1}{c}{ $j$ } & \multicolumn{1}{c}{ $m_j$ } \\
\hline
1     & 0   & 0   & 1/2 & -1/2  \\
2     & 0   & 0   & 1/2 & 1/2   \\
3     & 1   & 0   & 1/2 & -1/2  \\
4     & 1   & 0   & 1/2 & 1/2   \\
\hline
\end{tabular}
\end{quote}

\noindent
Note that the order is arbitrary.
There are $\left ( \begin{array}{c} 4 \\ 2 \end{array} \right) = 6$ two-particle states, 
which we list with the total $M$:


\begin{quote}
\begin{tabular}{cc}
\hline
\multicolumn{1}{c}{ Occupied } & \multicolumn{1}{c}{ $M$ } \\
\hline
1,2      & 0   \\
1,3      & -1  \\
1,4      & 0   \\
2,3      & 0   \\
2,4      & 1   \\
3,4      & 0   \\
\hline
\end{tabular}
\end{quote}

\noindent
and 1 each with $M = \pm 1$.




As another example, consider using only single particle states from the $0d_{5/2}$ space. 
They have the following quantum numbers


\begin{quote}
\begin{tabular}{ccccc}
\hline
\multicolumn{1}{c}{ Index } & \multicolumn{1}{c}{ $n$ } & \multicolumn{1}{c}{ $l$ } & \multicolumn{1}{c}{ $j$ } & \multicolumn{1}{c}{ $m_j$ } \\
\hline
1     & 0   & 2   & 5/2 & -5/2  \\
2     & 0   & 2   & 5/2 & -3/2  \\
3     & 0   & 2   & 5/2 & -1/2  \\
4     & 0   & 2   & 5/2 & 1/2   \\
5     & 0   & 2   & 5/2 & 3/2   \\
6     & 0   & 2   & 5/2 & 5/2   \\
\hline
\end{tabular}
\end{quote}

\noindent
There are $\left ( \begin{array}{c} 6 \\ 2 \end{array} \right) = 15$ two-particle states, 
which we list with the total $M$:


\begin{quote}
\begin{tabular}{cccccc}
\hline
\multicolumn{1}{c}{ Occupied } & \multicolumn{1}{c}{ $M$ } & \multicolumn{1}{c}{ Occupied } & \multicolumn{1}{c}{ $M$ } & \multicolumn{1}{c}{ Occupied } & \multicolumn{1}{c}{ $M$ } \\
\hline
1,2      & -4  & 2,3      & -2  & 3,5      & 1   \\
1,3      & -3  & 2,4      & -1  & 3,6      & 2   \\
1,4      & -2  & 2,5      & 0   & 4,5      & 2   \\
1,5      & -1  & 2,6      & 1   & 4,6      & 3   \\
1,6      & 0   & 3,4      & 0   & 5,6      & 4   \\
\hline
\end{tabular}
\end{quote}

\noindent




\subsection{Example case: pairing Hamiltonian, the warm-up project}


We consider a space with $2\Omega$ single-particle states, with each 
state labeled by 
$k = 1, 2, 3, \Omega$ and $m = \pm 1/2$. The convention is that 
the state with $k>0$ has $m = + 1/2$ while $-k$ has $m = -1/2$.


The Hamiltonian we consider is 
\[
\hat{H} = -\frac{g}{2} \hat{P}_+ \hat{P}_-,
\]
where
\[
\hat{P}_+ = \sum_{k > 0} \hat{a}^\dagger_k \hat{a}^\dagger_{-{k}}.
\]
and $\hat{P}_- = ( \hat{P}_+)^\dagger$.

This problem can be solved using what is called the quasi-spin formalism to obtain the 
exact results. Thereafter we will try again using the explicit Slater determinant formalism.


In the first part project we will consider four doubly degenerate single-particle states, resulting in eight single-particle states as shown here

\begin{quote}
\begin{tabular}{ccccc}
\hline
\multicolumn{1}{c}{ Index } & \multicolumn{1}{c}{ $n$ } & \multicolumn{1}{c}{ $l$ } & \multicolumn{1}{c}{ $s$ } & \multicolumn{1}{c}{ $m_s$ } \\
\hline
1     & 0   & 0   & 1/2 & -1/2  \\
2     & 0   & 0   & 1/2 & 1/2   \\
3     & 1   & 0   & 1/2 & -1/2  \\
4     & 1   & 0   & 1/2 & 1/2   \\
5     & 2   & 0   & 1/2 & -1/2  \\
6     & 2   & 0   & 1/2 & 1/2   \\
7     & 3   & 0   & 1/2 & -1/2  \\
8     & 3   & 0   & 1/2 & 1/2   \\
\hline
\end{tabular}
\end{quote}

\noindent

If we limit ourselves to four fermions only and states with no broken pairs, 
total $M=0$ states, we end with  sixSlater  determinants

\begin{itemize}
\item $|           1,           2 ,          3         ,       4  \rangle , $

\item $|            1      ,     2        ,        5         ,       6 \rangle , $

\item $|            1         ,       2     ,           7         ,       8  \rangle , $

\item $|            3        ,        4      ,          5          ,      6  \rangle , $

\item $|            3        ,        4      ,          7         ,       8  \rangle , $

\item $|            5        ,        6     ,           7     ,           8  \rangle $
\end{itemize}

\noindent
For our example, the $ 6 \times 6$  Hamiltonian matrix becomes
\[
H = \left ( 
\begin{array}{cccccc}
2\delta -g & -g/2 & -g/2 & -g/2 & -g/2 & 0 \\
 -g/2 & 4\delta -g & -g/2 & -g/2 & -0 & -g/2 \\
-g/2 & -g/2 & 6\delta -g & 0 & -g/2 & -g/2 \\
 -g/2 & -g/2 & 0 & 6\delta-g & -g/2 & -g/2 \\
 -g/2 & 0 & -g/2 & -g/2 & 8\delta-g & -g/2 \\
0 & -g/2 & -g/2 & -g/2 & -g/2 & 10\delta -g 
\end{array} \right )
\]
(You should check by hand that this is correct.) 

For $\delta = 0$ we have the closed form solution of  the g.s. energy given by $-6G$.

\section{Many-body perturbation theory}

\subsection{Many-body perturbation theory}

We assume here that we are only interested in the ground state of the system and 
expand the exact wave function in term of a series of Slater determinants
\[
\vert \Psi_0\rangle = \vert \Phi_0\rangle + \sum_{m=1}^{\infty}C_m\vert \Phi_m\rangle,
\]
where we have assumed that the true ground state is dominated by the 
solution of the unperturbed problem, that is
\[
\hat{H}_0\vert \Phi_0\rangle= W_0\vert \Phi_0\rangle.
\]
The state $\vert \Psi_0\rangle$ is not normalized, rather we have used an intermediate 
normalization $\langle \Phi_0 \vert \Psi_0\rangle=1$ since we have $\langle \Phi_0\vert \Phi_0\rangle=1$. 



The Schroedinger equation is
\[
\hat{H}\vert \Psi_0\rangle = E\vert \Psi_0\rangle,
\]
and multiplying the latter from the left with $\langle \Phi_0\vert $ gives
\[
\langle \Phi_0\vert \hat{H}\vert \Psi_0\rangle = E\langle \Phi_0\vert \Psi_0\rangle=E,
\]
and subtracting from this equation
\[
\langle \Psi_0\vert \hat{H}_0\vert \Phi_0\rangle= W_0\langle \Psi_0\vert \Phi_0\rangle=W_0,
\]
and using the fact that the both operators $\hat{H}$ and $\hat{H}_0$ are hermitian 
results in
\[
\Delta E=E-W_0=\langle \Phi_0\vert \hat{H}_I\vert \Psi_0\rangle,
\]
which is an exact result. We call this quantity the correlation energy.



This equation forms the starting point for all perturbative derivations. However,
as it stands it represents nothing but a mere formal rewriting of Schroedinger's equation and is not of much practical use. The exact wave function $\vert \Psi_0\rangle$ is unknown. In order to obtain a perturbative expansion, we need to expand the exact wave function in terms of the interaction $\hat{H}_I$. 

Here we have assumed that our model space defined by the operator $\hat{P}$ is one-dimensional, meaning that
\[
\hat{P}= \vert \Phi_0\rangle \langle \Phi_0\vert ,
\]
and
\[
\hat{Q}=\sum_{m=1}^{\infty}\vert \Phi_m\rangle \langle \Phi_m\vert .
\]


We can thus rewrite the exact wave function as
\[
\vert \Psi_0\rangle= (\hat{P}+\hat{Q})\vert \Psi_0\rangle=\vert \Phi_0\rangle+\hat{Q}\vert \Psi_0\rangle.
\]
Going back to the Schr\"odinger equation, we can rewrite it as, adding and a subtracting a term $\omega \vert \Psi_0\rangle$ as
\[
\left(\omega-\hat{H}_0\right)\vert \Psi_0\rangle=\left(\omega-E+\hat{H}_I\right)\vert \Psi_0\rangle,
\]
where $\omega$ is an energy variable to be specified later. 


We assume also that the resolvent of $\left(\omega-\hat{H}_0\right)$ exits, that is
it has an inverse which defined the unperturbed Green's function as
\[
\left(\omega-\hat{H}_0\right)^{-1}=\frac{1}{\left(\omega-\hat{H}_0\right)}.
\]

We can rewrite Schroedinger's equation as
\[
\vert \Psi_0\rangle=\frac{1}{\omega-\hat{H}_0}\left(\omega-E+\hat{H}_I\right)\vert \Psi_0\rangle,
\]
and multiplying from the left with $\hat{Q}$ results in
\[
\hat{Q}\vert \Psi_0\rangle=\frac{\hat{Q}}{\omega-\hat{H}_0}\left(\omega-E+\hat{H}_I\right)\vert \Psi_0\rangle,
\]
which is possible since we have defined the operator $\hat{Q}$ in terms of the eigenfunctions of $\hat{H}$.




These operators commute meaning that
\[
\hat{Q}\frac{1}{\left(\omega-\hat{H}_0\right)}\hat{Q}=\hat{Q}\frac{1}{\left(\omega-\hat{H}_0\right)}=\frac{\hat{Q}}{\left(\omega-\hat{H}_0\right)}.
\]
With these definitions we can in turn define the wave function as 
\[
\vert \Psi_0\rangle=\vert \Phi_0\rangle+\frac{\hat{Q}}{\omega-\hat{H}_0}\left(\omega-E+\hat{H}_I\right)\vert \Psi_0\rangle.
\]
This equation is again nothing but a formal rewrite of Schr\"odinger's equation
and does not represent a practical calculational scheme.  
It is a non-linear equation in two unknown quantities, the energy $E$ and the exact
wave function $\vert \Psi_0\rangle$. We can however start with a guess for $\vert \Psi_0\rangle$ on the right hand side of the last equation.



 The most common choice is to start with the function which is expected to exhibit the largest overlap with the wave function we are searching after, namely $\vert \Phi_0\rangle$. This can again be inserted in the solution for $\vert \Psi_0\rangle$ in an iterative fashion and if we continue along these lines we end up with
\[
\vert \Psi_0\rangle=\sum_{i=0}^{\infty}\left\{\frac{\hat{Q}}{\omega-\hat{H}_0}\left(\omega-E+\hat{H}_I\right)\right\}^i\vert \Phi_0\rangle, 
\]
for the wave function and
\[
\Delta E=\sum_{i=0}^{\infty}\langle \Phi_0\vert \hat{H}_I\left\{\frac{\hat{Q}}{\omega-\hat{H}_0}\left(\omega-E+\hat{H}_I\right)\right\}^i\vert \Phi_0\rangle, 
\]
which is now  a perturbative expansion of the exact energy in terms of the interaction
$\hat{H}_I$ and the unperturbed wave function $\vert \Psi_0\rangle$.



In our equations for $\vert \Psi_0\rangle$ and $\Delta E$ in terms of the unperturbed
solutions $\vert \Phi_i\rangle$  we have still an undetermined parameter $\omega$
and a dependecy on the exact energy $E$. Not much has been gained thus from a practical computational point of view. 

In Brilluoin-Wigner perturbation theory it is customary to set $\omega=E$. This results in the following perturbative expansion for the energy $\Delta E$
\[
\Delta E=\sum_{i=0}^{\infty}\langle \Phi_0\vert \hat{H}_I\left\{\frac{\hat{Q}}{\omega-\hat{H}_0}\left(\omega-E+\hat{H}_I\right)\right\}^i\vert \Phi_0\rangle=
\]
\[
\langle \Phi_0\vert \left(\hat{H}_I+\hat{H}_I\frac{\hat{Q}}{E-\hat{H}_0}\hat{H}_I+
\hat{H}_I\frac{\hat{Q}}{E-\hat{H}_0}\hat{H}_I\frac{\hat{Q}}{E-\hat{H}_0}\hat{H}_I+\dots\right)\vert \Phi_0\rangle. 
\]

\[
\Delta E=\sum_{i=0}^{\infty}\langle \Phi_0\vert \hat{H}_I\left\{\frac{\hat{Q}}{\omega-\hat{H}_0}\left(\omega-E+\hat{H}_I\right)\right\}^i\vert \Phi_0\rangle=\]
\[
\langle \Phi_0\vert \left(\hat{H}_I+\hat{H}_I\frac{\hat{Q}}{E-\hat{H}_0}\hat{H}_I+
\hat{H}_I\frac{\hat{Q}}{E-\hat{H}_0}\hat{H}_I\frac{\hat{Q}}{E-\hat{H}_0}\hat{H}_I+\dots\right)\vert \Phi_0\rangle. 
\]
This expression depends however on the exact energy $E$ and is again not very convenient from a practical point of view. It can obviously be solved iteratively, by starting with a guess for  $E$ and then solve till some kind of self-consistency criterion has been reached. 

Actually, the above expression is nothing but a rewrite again of the full Schr\"odinger equation. 

Defining $e=E-\hat{H}_0$ and recalling that $\hat{H}_0$ commutes with 
$\hat{Q}$ by construction and that $\hat{Q}$ is an idempotent operator
$\hat{Q}^2=\hat{Q}$. 
Using this equation in the above expansion for $\Delta E$ we can write the denominator 
\[
\hat{Q}\frac{1}{\hat{e}-\hat{Q}\hat{H}_I\hat{Q}}=
\]
\[
\hat{Q}\left[\frac{1}{\hat{e}}+\frac{1}{\hat{e}}\hat{Q}\hat{H}_I\hat{Q}
\frac{1}{\hat{e}}+\frac{1}{\hat{e}}\hat{Q}\hat{H}_I\hat{Q}
\frac{1}{\hat{e}}\hat{Q}\hat{H}_I\hat{Q}\frac{1}{\hat{e}}+\dots\right]\hat{Q}.
\]

Inserted in the expression for $\Delta E$ leads to 
\[
\Delta E=
\langle \Phi_0\vert \hat{H}_I+\hat{H}_I\hat{Q}\frac{1}{E-\hat{H}_0-\hat{Q}\hat{H}_I\hat{Q}}\hat{Q}\hat{H}_I\vert \Phi_0\rangle. 
\]
In RS perturbation theory we set $\omega = W_0$ and obtain the following expression for the energy difference
\[
\Delta E=\sum_{i=0}^{\infty}\langle \Phi_0\vert \hat{H}_I\left\{\frac{\hat{Q}}{W_0-\hat{H}_0}\left(\hat{H}_I-\Delta E\right)\right\}^i\vert \Phi_0\rangle=
\]
\[
\langle \Phi_0\vert \left(\hat{H}_I+\hat{H}_I\frac{\hat{Q}}{W_0-\hat{H}_0}(\hat{H}_I-\Delta E)+
\hat{H}_I\frac{\hat{Q}}{W_0-\hat{H}_0}(\hat{H}_I-\Delta E)\frac{\hat{Q}}{W_0-\hat{H}_0}(\hat{H}_I-\Delta E)+\dots\right)\vert \Phi_0\rangle.
\]



Recalling that $\hat{Q}$ commutes with $\hat{H_0}$ and since $\Delta E$ is a constant we obtain that
\[
\hat{Q}\Delta E\vert \Phi_0\rangle = \hat{Q}\Delta E\vert \hat{Q}\Phi_0\rangle = 0.
\]
Inserting this results in the expression for the energy results in
\[
\Delta E=\langle \Phi_0\vert \left(\hat{H}_I+\hat{H}_I\frac{\hat{Q}}{W_0-\hat{H}_0}\hat{H}_I+
\hat{H}_I\frac{\hat{Q}}{W_0-\hat{H}_0}(\hat{H}_I-\Delta E)\frac{\hat{Q}}{W_0-\hat{H}_0}\hat{H}_I+\dots\right)\vert \Phi_0\rangle.
\]



We can now this expression in terms of a perturbative expression in terms
of $\hat{H}_I$ where we iterate the last expression in terms of $\Delta E$
\[
\Delta E=\sum_{i=1}^{\infty}\Delta E^{(i)}.
\]
We get the following expression for $\Delta E^{(i)}$
\[
\Delta E^{(1)}=\langle \Phi_0\vert \hat{H}_I\vert \Phi_0\rangle,
\] 
which is just the contribution to first order in perturbation theory,
\[
\Delta E^{(2)}=\langle\Phi_0\vert \hat{H}_I\frac{\hat{Q}}{W_0-\hat{H}_0}\hat{H}_I\vert \Phi_0\rangle, 
\]
which is the contribution to second order.



\[
\Delta E^{(3)}=\langle \Phi_0\vert \hat{H}_I\frac{\hat{Q}}{W_0-\hat{H}_0}\hat{H}_I\frac{\hat{Q}}{W_0-\hat{H}_0}\hat{H}_I\Phi_0\rangle-
\langle\Phi_0\vert \hat{H}_I\frac{\hat{Q}}{W_0-\hat{H}_0}\langle \Phi_0\vert \hat{H}_I\vert \Phi_0\rangle\frac{\hat{Q}}{W_0-\hat{H}_0}\hat{H}_I\vert \Phi_0\rangle,
\]
being the third-order contribution. 


\subsection{Interpreting the correlation energy and the wave operator}

In the shell-model lectures we showed that we could rewrite the exact state function for say the ground state, as a linear expansion in terms of all possible Slater determinants. That is, we 
define the ansatz for the ground state as 
\[
|\Phi_0\rangle = \left(\prod_{i\le F}\hat{a}_{i}^{\dagger}\right)|0\rangle,
\]
where the index $i$ defines different single-particle states up to the Fermi level. We have assumed that we have $N$ fermions. 
A given one-particle-one-hole ($1p1h$) state can be written as
\[
|\Phi_i^a\rangle = \hat{a}_{a}^{\dagger}\hat{a}_i|\Phi_0\rangle,
\]
while a $2p2h$ state can be written as
\[
|\Phi_{ij}^{ab}\rangle = \hat{a}_{a}^{\dagger}\hat{a}_{b}^{\dagger}\hat{a}_j\hat{a}_i|\Phi_0\rangle,
\]
and a general $ApAh$ state as 
\[
|\Phi_{ijk\dots}^{abc\dots}\rangle = \hat{a}_{a}^{\dagger}\hat{a}_{b}^{\dagger}\hat{a}_{c}^{\dagger}\dots\hat{a}_k\hat{a}_j\hat{a}_i|\Phi_0\rangle.
\]

We use letters $ijkl\dots$ for states below the Fermi level and $abcd\dots$ for states above the Fermi level. A general single-particle state is given by letters $pqrs\dots$.

We can then expand our exact state function for the ground state 
as
\[
|\Psi_0\rangle=C_0|\Phi_0\rangle+\sum_{ai}C_i^a|\Phi_i^a\rangle+\sum_{abij}C_{ij}^{ab}|\Phi_{ij}^{ab}\rangle+\dots
=(C_0+\hat{C})|\Phi_0\rangle,
\]
where we have introduced the so-called correlation operator 
\[
\hat{C}=\sum_{ai}C_i^a\hat{a}_{a}^{\dagger}\hat{a}_i  +\sum_{abij}C_{ij}^{ab}\hat{a}_{a}^{\dagger}\hat{a}_{b}^{\dagger}\hat{a}_j\hat{a}_i+\dots
\]
Since the normalization of $\Psi_0$ is at our disposal and since $C_0$ is by hypothesis non-zero, we may arbitrarily set $C_0=1$ with 
corresponding proportional changes in all other coefficients. Using this so-called intermediate normalization we have
\[
\langle \Psi_0 | \Phi_0 \rangle = \langle \Phi_0 | \Phi_0 \rangle = 1, 
\]
resulting in 
\[
|\Psi_0\rangle=(1+\hat{C})|\Phi_0\rangle.
\]

In a shell-model calculation, the unknown coefficients in $\hat{C}$ are the 
eigenvectors which result from the diagonalization of the Hamiltonian matrix.

How can we use perturbation theory to determine the same coefficients? Let us study the contributions to second order in the interaction, namely
\[
\Delta E^{(2)}=\langle\Phi_0\vert \hat{H}_I\frac{\hat{Q}}{W_0-\hat{H}_0}\hat{H}_I\vert \Phi_0\rangle.
\]

The intermediate states given by $\hat{Q}$ can at most be of a $2p-2h$ nature if we have a two-body Hamiltonian. This means that second order in the perturbation theory can have $1p-1h$ and $2p-2h$ at most as intermediate states. When we diagonalize, these contributions are included to infinite order. This means that higher-orders in perturbation theory bring in more complicated correlations. 

If we limit the attention to a Hartree-Fock basis, then we have that
$\langle\Phi_0\vert \hat{H}_I \vert 2p-2h\rangle$ is the only contribution and the contribution to the energy reduces to
\[
\Delta E^{(2)}=\frac{1}{4}\sum_{abij}\langle ij\vert \hat{v}\vert ab\rangle \frac{\langle ab\vert \hat{v}\vert ij\rangle}{\epsilon_i+\epsilon_j-\epsilon_a-\epsilon_b}.
\]

If we compare this to the correlation energy obtained from full configuration interaction theory with a Hartree-Fock basis, we found that
\[
E-E_0 =\Delta E=
\sum_{abij}\langle ij | \hat{v}| ab \rangle C_{ij}^{ab},
\]
where the energy $E_0$ is the reference energy and $\Delta E$ defines the so-called correlation energy.

We see that if we set
\[
C_{ij}^{ab} =\frac{1}{4}\frac{\langle ab \vert \hat{v} \vert ij \rangle}{\epsilon_i+\epsilon_j-\epsilon_a-\epsilon_b},
\]
we have a perfect agreement between FCI and MBPT. However, FCI includes such $2p-2h$ correlations to infinite order. In order to make a meaningful comparison we would at least need to sum such correlations to infinite order in perturbation theory. 

Summing up, we can see that
\begin{itemize}
\item MBPT introduces order-by-order specific correlations and we make comparisons with exact calculations like FCI

\item At every order, we can calculate all contributions since they are well-known and either tabulated or calculated on the fly.

\item MBPT is a non-variational theory and there is no guarantee that higher orders will improve the convergence. 

\item However, since FCI calculations are limited by the size of the Hamiltonian matrices to diagonalize (today's most efficient codes can attach dimensionalities of ten billion basis states, MBPT can function as an approximative method which gives a straightforward (but tedious) calculation recipe. 

\item MBPT has been widely used to compute effective interactions for the nuclear shell-model.

\item But there are better methods which sum to infinite order important correlations. Coupled cluster theory is one of these methods. 
\end{itemize}

\section{Coupled cluster theory}
\section{Introduction}
Coester and Kummel first developed the ideas that led to coupled-cluster
theory in the late 1950s. The basic idea is that the correlated wave function
of a many-body system $\mid\Psi\rangle$
can be formulated as an exponential of correlation
operators $T$ acting on a reference state $\mid\Phi\rangle$
\[
\mid\Psi\rangle = \exp\left(-\hat{T}\right)\mid\Phi\rangle\ .
\]
We will discuss how to define the operators later in this work. This simple
ansatz carries enormous power. It leads to a non-perturbative many-body
theory that includes summation of ladder diagrams , ring
diagrams, and an infinite-order
generalization of many-body perturbation theory..

Developments and applications
of coupled-cluster theory took different routes in chemistry
and nuclear physics. In quantum chemistry,
coupled-cluster developments
and applications have proven to be extremely useful, see for example the review by \href{{http://journals.aps.org/rmp/abstract/10.1103/RevModPhys.79.291}}{Barrett and Musial} as well as the recent 
textbook by \href{{http://www.cambridge.org/fr/academic/subjects/chemistry/physical-chemistry/many-body-methods-chemistry-and-physics-mbpt-and-coupled-cluster-theory?format=HB}}{Shavitt and Barrett}.  Many previous applications to nuclear physics struggled with the repulsive character of the nuclear forces and limited basis sets used in the computations. Most of these problems have been overcome during the last decade and coupled-cluster
theory is one of the computational methods of preference for doing nuclear physics, with applications ranging from light nuclei to medium-heavy nuclei,
see for example the recent review by \href{{http://iopscience.iop.org/0034-4885/77/9/096302}}{Hagen, Papenbrock, Hjorth-Jensen and Dean}. 


\subsection{A non-practical way of solving the eigenvalue problem}

Before we proceed with the derivation of the Coupled cluster equations, let us repeat some of the arguments we presented during our FCI lectures. 
In our FCI discussions, we rewrote the solution of the Schroedinger equation as a set of coupled equationsin the unknown coefficients $C$. Let us repeat some of these arguments.
To obtain the eigenstates and eigenvalues in terms of non-linear equations is not a very practical approach. However, it serves the scope of linking FCI theory with approximative solutions to the many-body problem
like Coupled cluster (CC) theory 

If we assume that we have a two-body operator at most, the Slater-Condon rule 
gives then an equation for the 
correlation energy in terms of $C_i^a$ and $C_{ij}^{ab}$ only.  We get then
\[
\langle \Phi_0 | \hat{H} -E| \Phi_0\rangle + \sum_{ai}\langle \Phi_0 | \hat{H} -E|\Phi_{i}^{a} \rangle C_{i}^{a}+
\sum_{abij}\langle \Phi_0 | \hat{H} -E|\Phi_{ij}^{ab} \rangle C_{ij}^{ab}=0,
\]
or 
\[
E-E_0 =\Delta E=\sum_{ai}\langle \Phi_0 | \hat{H}|\Phi_{i}^{a} \rangle C_{i}^{a}+
\sum_{abij}\langle \Phi_0 | \hat{H}|\Phi_{ij}^{ab} \rangle C_{ij}^{ab},
\]
where the energy $E_0$ is the reference energy and $\Delta E$ defines the so-called correlation energy.
The single-particle basis functions  could be the results of a Hartree-Fock calculation or just the eigenstates of the non-interacting part of the Hamiltonian. 

In our notes on Hartree-Fock calculations, 
we have already computed the matrix $\langle \Phi_0 | \hat{H}|\Phi_{i}^{a}\rangle $ and $\langle \Phi_0 | \hat{H}|\Phi_{ij}^{ab}\rangle$.  If we are using a Hartree-Fock basis, then the matrix elements
$\langle \Phi_0 | \hat{H}|\Phi_{i}^{a}\rangle=0$ and we are left with a \emph{correlation energy} given by
\[
E-E_0 =\Delta E^{HF}=\sum_{abij}\langle \Phi_0 | \hat{H}|\Phi_{ij}^{ab} \rangle C_{ij}^{ab}. 
\]


Inserting the various matrix elements we can rewrite the previous equation as
\[
\Delta E=\sum_{ai}\langle i| \hat{f}|a \rangle C_{i}^{a}+
\sum_{abij}\langle ij | \hat{v}| ab \rangle C_{ij}^{ab}.
\]
This equation determines the correlation energy but not the coefficients $C$. 
We need more equations. Our next step is to set up
\[
\langle \Phi_i^a | \hat{H} -E| \Phi_0\rangle + \sum_{bj}\langle \Phi_i^a | \hat{H} -E|\Phi_{j}^{b} \rangle C_{j}^{b}+
\sum_{bcjk}\langle \Phi_i^a | \hat{H} -E|\Phi_{jk}^{bc} \rangle C_{jk}^{bc}+
\sum_{bcdjkl}\langle \Phi_i^a | \hat{H} -E|\Phi_{jkl}^{bcd} \rangle C_{jkl}^{bcd}=0,
\]
as this equation will allow us to find an expression for the coefficents $C_i^a$ since we can rewrite this equation as 
\[
\langle i | \hat{f}| a\rangle +\langle \Phi_i^a | \hat{H}|\Phi_{i}^{a} \rangle C_{i}^{a}+ \sum_{bj\ne ai}\langle \Phi_i^a | \hat{H}|\Phi_{j}^{b} \rangle C_{j}^{b}+
\sum_{bcjk}\langle \Phi_i^a | \hat{H}|\Phi_{jk}^{bc} \rangle C_{jk}^{bc}+
\sum_{bcdjkl}\langle \Phi_i^a | \hat{H}|\Phi_{jkl}^{bcd} \rangle C_{jkl}^{bcd}=EC_i^a.
\]

We see that on the right-hand side we have the energy $E$. This leads to a non-linear equation in the unknown coefficients. 
These equations are normally solved iteratively ( that is we can start with a guess for the coefficients $C_i^a$). A common choice is to use perturbation theory for the first guess, setting thereby
\[
 C_{i}^{a}=\frac{\langle i | \hat{f}| a\rangle}{\epsilon_i-\epsilon_a}.
\]

The observant reader will however see that we need an equation for $C_{jk}^{bc}$ and $C_{jkl}^{bcd}$ as well.
To find equations for these coefficients we need then to continue our multiplications from the left with the various
$\Phi_{H}^P$ terms. 


For $C_{jk}^{bc}$ we need then
\[
\langle \Phi_{ij}^{ab} | \hat{H} -E| \Phi_0\rangle + \sum_{kc}\langle \Phi_{ij}^{ab} | \hat{H} -E|\Phi_{k}^{c} \rangle C_{k}^{c}+
\]
\[
\sum_{cdkl}\langle \Phi_{ij}^{ab} | \hat{H} -E|\Phi_{kl}^{cd} \rangle C_{kl}^{cd}+\sum_{cdeklm}\langle \Phi_{ij}^{ab} | \hat{H} -E|\Phi_{klm}^{cde} \rangle C_{klm}^{cde}+\sum_{cdefklmn}\langle \Phi_{ij}^{ab} | \hat{H} -E|\Phi_{klmn}^{cdef} \rangle C_{klmn}^{cdef}=0,
\]
and we can isolate the coefficients $C_{kl}^{cd}$ in a similar way as we did for the coefficients $C_{i}^{a}$. 
A standard choice for the first iteration is to set 
\[
C_{ij}^{ab} =\frac{\langle ij \vert \hat{v} \vert ab \rangle}{\epsilon_i+\epsilon_j-\epsilon_a-\epsilon_b}.
\]
At the end we can rewrite our solution of the Schroedinger equation in terms of $n$ coupled equations for the coefficients $C_H^P$.
This is a very cumbersome way of solving the equation. However, by using this iterative scheme we can illustrate how we can compute the
various terms in the wave operator or correlation operator $\hat{C}$. We will later identify the calculation of the various terms $C_H^P$
as parts of different many-body approximations to full CI. In particular, we can  relate this non-linear scheme with Coupled Cluster theory and
many-body perturbation theory.


\subsection{Summarizing FCI and bringing in approximative methods}


If we can diagonalize large matrices, FCI is the method of choice since:
\begin{itemize}
\item It gives all eigenvalues, ground state and excited states

\item The eigenvectors are obtained directly from the coefficients $C_H^P$ which result from the diagonalization

\item We can compute easily expectation values of other operators, as well as transition probabilities

\item Correlations are easy to understand in terms of contributions to a given operator beyond the Hartree-Fock contribution. This is the standard approach in  many-body theory. 
\end{itemize}

\noindent
The correlation energy is defined as, with a two-body Hamiltonian,  
\[
\Delta E=\sum_{ai}\langle i| \hat{f}|a \rangle C_{i}^{a}+
\sum_{abij}\langle ij | \hat{v}| ab \rangle C_{ij}^{ab}.
\]

The coefficients $C$ result from the solution of the eigenvalue problem. 
The energy of say the ground state is then
\[
E=E_{ref}+\Delta E,
\]
where the so-called reference energy is the energy we obtain from a Hartree-Fock calculation, that is
\[
E_{ref}=\langle \Phi_0 \vert \hat{H} \vert \Phi_0 \rangle.
\]

However, as we have seen, even for a small case like the four first major shells and a nucleus like oxygen-16, the dimensionality becomes quickly intractable. If we wish to include single-particle states that reflect weakly bound systems, we need a much larger single-particle basis. We need thus approximative methods that sum specific correlations to infinite order. 

Popular methods are
\begin{itemize}
\item \href{{http://www.sciencedirect.com/science/journal/03701573/261/3-4}}{Many-body perturbation theory (in essence a Taylor expansion)}

\item \href{{http://iopscience.iop.org/0034-4885/77/9/096302}}{Coupled cluster theory (coupled non-linear equations)}

\item \href{{http://www.worldscientific.com/worldscibooks/10.1142/6821}}{Green's function approaches (matrix inversion)}

\item \href{{http://journals.aps.org/prc/abstract/10.1103/PhysRevC.85.061304}}{Similarity group transformation methods (coupled ordinary differential equations)}
\end{itemize}

\noindent
All these methods start normally with a Hartree-Fock basis as the calculational basis. 


\subsection{A quick tour of Coupled Cluster theory}

The ansatz for the wavefunction (ground state) is given by
\begin{equation*}
   \vert \Psi\rangle = \vert \Psi_{CC}\rangle = e^{\hat{T}} \vert \Phi_0\rangle =  
  \left( \sum_{n=1}^{A} \frac{1}{n!} \hat{T}^n \right) \vert \Phi_0\rangle,
\end{equation*}
where $A$ represents the maximum number of particle-hole excitations and $\hat{T}$ is the cluster operator defined as
\begin{align*}
            \hat{T} &= \hat{T}_1 + \hat{T}_2 + \ldots + \hat{T}_A \\
            \hat{T}_n &= \left(\frac{1}{n!}\right)^2 
                \sum_{\substack{
                        i_1,i_2,\ldots i_n \\
                        a_1,a_2,\ldots a_n}}
                t_{i_1i_2\ldots i_n}^{a_1a_2\ldots a_n} a_{a_1}^\dagger a_{a_2}^\dagger \ldots a_{a_n}^\dagger a_{i_n} \ldots a_{i_2} a_{i_1}.
        \end{align*}
    The energy is given by
    \begin{equation*}
        E_{\mathrm{CC}} = \langle\Phi_0\vert  \overline{H}\vert \Phi_0\rangle,
    \end{equation*}
    where $\overline{H}$ is a similarity transformed Hamiltonian
    \begin{align*}
        \overline{H}&= e^{-\hat{T}} \hat{H}_N e^{\hat{T}} \\
        \hat{H}_N &= \hat{H} - \langle\Phi_0\vert \hat{H} \vert \Phi_0\rangle.
    \end{align*}

    The coupled cluster energy is a function of the unknown cluster amplitudes $t_{i_1i_2\ldots i_n}^{a_1a_2\ldots a_n}$,
given by the solutions to the amplitude equations
    \begin{equation*}
        0 = \langle\Phi_{i_1 \ldots i_n}^{a_1 \ldots a_n}\vert \overline{H}\vert \Phi_0\rangle.
    \end{equation*}
The similarity transformed   Hamiltonian  $\overline{H}$ is expanded using the Baker-Campbell-Hausdorff expression,
    \begin{align*}
        \overline{H}&= \hat{H}_N + \left[ \hat{H}_N, \hat{T} \right] + 
            \frac{1}{2} \left[\left[ \hat{H}_N, \hat{T} \right], \hat{T}\right] + \ldots \\
            & \quad \frac{1}{n!} \left[ \ldots \left[ \hat{H}_N, \hat{T} \right], \ldots \hat{T} \right] +\dots
    \end{align*}
and simplified using the connected cluster theorem
    \begin{equation*}
        \overline{H}= \hat{H}_N + \left( \hat{H}_N \hat{T}\right)_c + \frac{1}{2} \left( \hat{H}_N \hat{T}^2\right)_c
            + \dots + \frac{1}{n!} \left( \hat{H}_N \hat{T}^n\right)_c +\dots
    \end{equation*}

A much used approximation is to  truncate the cluster operator $\hat{T}$ at the $n=2$ level. This defines the so-called singes and doubles approximation to the Coupled Cluster wavefunction, normally shortened to CCSD..

The coupled cluster wavefunction is now given by
\begin{equation*}
            \vert \Psi_{CC}\rangle = e^{\hat{T}_1 + \hat{T}_2} \vert \Phi_0\rangle
\end{equation*}
where 
        \begin{align*}
            \hat{T}_1 &= 
            \sum_{ia}
                t_{i}^{a} a_{a}^\dagger a_i \\
            \hat{T}_2 &= \frac{1}{4} 
            \sum_{ijab}
                t_{ij}^{ab} a_{a}^\dagger a_{b}^\dagger a_{j} a_{i}.
        \end{align*}

The amplutudes $t$ play a role similar to the coefficients $C$ in the shell-model calculations. They are obtained by solving a set of non-linear equations
similar to those discussed above in connection withe FCI discussion.

If we truncate our equations at the CCSD level, it corresponds to performing a transformation of the Hamiltonian matrix of the following type for a six particle problem (with a two-body Hamiltonian):


\begin{quote}
\begin{tabular}{cccccccc}
\hline
\multicolumn{1}{c}{  } & \multicolumn{1}{c}{ $0p-0h$ } & \multicolumn{1}{c}{ $1p-1h$ } & \multicolumn{1}{c}{ $2p-2h$ } & \multicolumn{1}{c}{ $3p-3h$ } & \multicolumn{1}{c}{ $4p-4h$ } & \multicolumn{1}{c}{ $5p-5h$ } & \multicolumn{1}{c}{ $6p-6h$ } \\
\hline
$0p-0h$ & $\tilde{x}$ & $\tilde{x}$ & $\tilde{x}$ & 0           & 0           & 0           & 0           \\
$1p-1h$ & 0           & $\tilde{x}$ & $\tilde{x}$ & $\tilde{x}$ & 0           & 0           & 0           \\
$2p-2h$ & 0           & $\tilde{x}$ & $\tilde{x}$ & $\tilde{x}$ & $\tilde{x}$ & 0           & 0           \\
$3p-3h$ & 0           & $\tilde{x}$ & $\tilde{x}$ & $\tilde{x}$ & $\tilde{x}$ & $\tilde{x}$ & 0           \\
$4p-4h$ & 0           & 0           & $\tilde{x}$ & $\tilde{x}$ & $\tilde{x}$ & $\tilde{x}$ & $\tilde{x}$ \\
$5p-5h$ & 0           & 0           & 0           & $\tilde{x}$ & $\tilde{x}$ & $\tilde{x}$ & $\tilde{x}$ \\
$6p-6h$ & 0           & 0           & 0           & 0           & $\tilde{x}$ & $\tilde{x}$ & $\tilde{x}$ \\
\hline
\end{tabular}
\end{quote}

\noindent


In our FCI discussion the correlation energy is defined as, with a two-body Hamiltonian,  
\[
\Delta E=\sum_{ai}\langle i| \hat{f}|a \rangle C_{i}^{a}+
\sum_{abij}\langle ij | \hat{v}| ab \rangle C_{ij}^{ab}.
\]

In Coupled cluster theory it becomes (irrespective of level of truncation of $T$)
\[
\Delta E=\sum_{ai}\langle i| \hat{f}|a \rangle t_{i}^{a}+
\sum_{abij}\langle ij | \hat{v}| ab \rangle t_{ij}^{ab}.
\]

Coupled cluster theory has several interesting computational features and is the method of choice in quantum chemistry. The method was originally proposed by Coester and Kummel, two nuclear physicists (way back in the fifties). It came back in full strength in nuclear physics during the last decade. 

There are several interesting features:
\begin{itemize}
\item With a truncation like CCSD or CCSDT, we can include to infinite order correlations like $2p-2h$.

\item We can include a large basis of single-particle states, not possible in standard FCI calculations
\end{itemize}

\noindent
However, Coupled Cluster theory is
\begin{itemize}
\item non-variational

\item if we want to find properties of excited states, additional calculations via for example equation of motion methods are needed

\item if correlations are strong, a single-reference ansatz may not be the best starting point

\item we cannot quantify properly the error we make when truncations are made in the cluster operator
\end{itemize}

\noindent
\subsection{The CCD approximation}

We will now approximate the cluster operator $\hat{T}$ to include only $2p-2h$ correlations. This leads to the so-called CCD approximation, that is
\[
\hat{T}\approx \hat{T}_2=\frac{1}{4}\sum_{abij}t_{ij}^{ab}a^{\dagger}_aa^{\dagger}_ba_ja_i,
\]
meaning that we have
\[
\vert \Psi_0 \rangle \approx \vert \Psi_{CCD} \rangle = \exp{\left(\hat{T}_2\right)}\vert \Phi_0\rangle.
\]

Inserting these equations in the expression for the computation of the energy we have,
with a Hamiltonian defined with respect to a general vacuum (see the exercises in the second quantization part)
\[
\hat{H}=\hat{H}_N+E_{\mathrm{ref}},
\]
with 
\[
\hat{H}_N=\sum_{pq}\langle p \vert \hat{f} \vert q \rangle  a^{\dagger}_pa_q + \frac{1}{4}\sum_{pqrs}\langle pq \vert \hat{v} \vert rs \rangle a^{\dagger}_pa^{\dagger}_qa_sa_r,
\]
we obtain that the energy can be written as 
\[
\langle \Phi_0 \vert \exp{-\left(\hat{T}_2\right)}\hat{H}_N\exp{\left(\hat{T}_2\right)}\vert \Phi_0\rangle =
\langle \Phi_0 \vert \hat{H}_N(1+\hat{T}_2)\vert \Phi_0\rangle = E_{CCD}.
\]
This quantity becomes 
\[
E_{CCD}=E_{\mathrm{ref}}+\frac{1}{4}\sum_{abij}\langle ij \vert \hat{v} \vert ab \rangle t_{ij}^{ab},
\]
where the latter is the correlation energy from this level of approximation of CC theory. 
Similarly, the expression for the amplitudes reads
\[
\langle \Phi_{ij}^{ab} \vert \exp{-\left(\hat{T}_2\right)}\hat{H}_N\exp{\left(\hat{T}_2\right)}\vert \Phi_0\rangle = 0.
\]
These equations can be reduced to (after several applications of Wick's theorem) to, for all $i > j$ and all $a  > b$,
\begin{align}
0 = \langle ab \vert \hat{v} \vert ij \rangle + \left(\epsilon_a+\epsilon_b-\epsilon_i-\epsilon_j\right)t_{ij}^{ab} & \nonumber \\ 
+\frac{1}{2}\sum_{cd} \langle ab \vert \hat{v} \vert cd \rangle t_{ij}^{cd}+\frac{1}{2}\sum_{kl} \langle kl \vert \hat{v} \vert ij \rangle t_{kl}^{ab}+\hat{P}(ij\vert ab)\sum_{kc} \langle kb \vert \hat{v} \vert cj \rangle t_{ik}^{ac} & \nonumber \\
+\frac{1}{4}\sum_{klcd} \langle kl \vert \hat{v} \vert cd \rangle t_{ij}^{cd}t_{kl}^{ab}+\hat{P}(ij)\sum_{klcd} \langle kl \vert \hat{v} \vert cd \rangle t_{ik}^{ac}t_{jl}^{bd}& \nonumber \\
-\frac{1}{2}\hat{P}(ij)\sum_{klcd} \langle kl \vert \hat{v} \vert cd \rangle t_{ik}^{dc}t_{lj}^{ab}-\frac{1}{2}\hat{P}(ab)\sum_{klcd} \langle kl \vert \hat{v} \vert cd \rangle t_{lk}^{ac}t_{ij}^{db},&
\label{eq:ccd}
\end{align}
where we have defined 
\[
\hat{P}\left(ab\right)= 1-\hat{P}_{ab},
\]
where $\hat{P}_{ab}$ interchanges two particles occupying the quantum numbers $a$ and $b$. 
The operator $\hat{P}(ij\vert ab)$  is defined as
\[
\hat{P}(ij\vert ab) = (1-\hat{P}_{ij})(1-\hat{P}_{ab}).
\]
Recall also that the unknown amplitudes $t_{ij}^{ab}$
represent anti-symmetrized matrix elements, meaning that they obey the same symmetry relations as the two-body interaction, that is
\[
t_{ij}^{ab}=-t_{ji}^{ab}=-t_{ij}^{ba}=t_{ji}^{ba}.
\]
The two-body matrix elements are also anti-symmetrized, meaning that
\[
\langle ab \vert \hat{v} \vert ij \rangle = -\langle ab \vert \hat{v} \vert ji \rangle= -\langle ba \vert \hat{v} \vert ij \rangle=\langle ba \vert \hat{v} \vert ji \rangle.
\]
The non-linear equations for the unknown amplitudes  $t_{ij}^{ab}$ are solved iteratively. We discuss the implementation of these equations below.

\paragraph{Approximations to the full CCD equations.}
It is useful to make approximations to the equations for the amplitudes. The standard method for solving these equations is to set up an iterative scheme where method's like Newton's method or similar root searching methods are used to find the amplitudes. 
Itreative solvers need a guess for the amplitudes. A good starting point is to use the correlated wave operator from perturbation theory to
first order in the interaction.
This means that we define the zeroth approximation to the amplitudes as 
\[
t^{(0)}=\frac{\langle ab \vert \hat{v} \vert ij \rangle}{\left(\epsilon_i+\epsilon_j-\epsilon_a-\epsilon_b\right)},
\]
leading to our first approximation for the correlation energy at the CCD level to be equal to second-order perturbation theory without $1p-1h$ excitations, namely
\[
\Delta E_{\mathrm{CCD}}^{(0)}=\frac{1}{4}\sum_{abij} \frac{\langle ij \vert \hat{v} \vert ab \rangle \langle ab \vert \hat{v} \vert ij \rangle}{\left(\epsilon_i+\epsilon_j-\epsilon_a-\epsilon_b\right)}.
\]

With this starting point, we are now ready to solve Eq. (\ref{eq:ccd}) iteratively. Before we attack the full equations, it is however instructive to study a truncated version of the equations. We will first study the following approximation where we take away all terms except the linear terms that involve the single-particle energies and the the two-particle intermediate excitations, that is
\begin{equation}
0 = \langle ab \vert \hat{v} \vert ij \rangle + \left(\epsilon_a+\epsilon_b-\epsilon_i-\epsilon_j\right)t_{ij}^{ab}+\frac{1}{2}\sum_{cd} \langle ab \vert \hat{v} \vert cd \rangle t_{ij}^{cd}.
\label{eq:ccd1}
\end{equation}

Setting the single-particle energies for the hole states equal to an energy variable $\omega = \epsilon_i+\epsilon_j$, Eq. (\ref{eq:ccd1}) reduces to the
well-known equations for the so-called $G$-matrix, widely used in \href{{http://www.sciencedirect.com/science/journal/03701573/261/3-4}}{infinite matter and finite nuclei studies}. The equation can then be reordered and solved by matrix inversion.  To see this let us define the following quantity
\[
\tau_{ij}^{ab}= \left(\omega-\epsilon_a-\epsilon_b\right)t_{ij}^{ab},
\]
and inserting 
\[
1=\frac{\left(\omega-\epsilon_c-\epsilon_d\right)}{\left(\omega-\epsilon_c-\epsilon_d\right)},
\]
in the intermediate sums over $cd$ in Eq. (\ref{eq:ccd1}), we can rewrite the latter equation as
\[
\tau_{ij}^{ab}(\omega)= \langle ab \vert \hat{v} \vert ij \rangle + \frac{1}{2}\sum_{cd} \langle ab \vert \hat{v} \vert cd \rangle \frac{1}{\omega-\epsilon_c-\epsilon_d}\tau_{ij}^{cd}(\omega),
\]
where we have indicated an explicit energy dependence. This equation, transforming a two-particle configuration into a single index, can be transformed into a matrix inversion problem.  Solving the equations for a fixed energy $\omega$ allows us to compare directly with results from Green's function theory when only two-particle intermediate states are included. 

To solve Eq. (\ref{eq:ccd1}), we would thus start with a guess for the unknown amplitudes, typically using the wave operator defined by first order in perturbation theory, leading to a zeroth approximation to the energy given by second-order perturbation theory for the correlation energy.
A simple approach to the solution of  Eq. (\ref{eq:ccd1}), is to thus to
\begin{enumerate}
\item Start with a guess for the amplitudes and compute the zeroth approximation to the correlation energy

\item Use the ansatz for the amplitudes to solve Eq. (\ref{eq:ccd1}) via for example your root-finding method of choice (Newton's method or modifications thereof can be used) and continue these iterations till the correlation energy does not change more than a prefixed quantity $\lambda$; $\Delta E_{\mathrm{CCD}}^{(i)}-\Delta E_{\mathrm{CCD}}^{(i-1)} \le \lambda$.

\item It is common during the iterations to scale the amplitudes with a parameter $\alpha$, with $\alpha \in (0,1]$ as  $t^{(i)}=\alpha t^{(i)}+(1-\alpha)t^{(i-1)}$.
\end{enumerate}

\noindent
The next approximation is to include the two-hole term in Eq. (\ref{eq:ccd}), a term which allow us to make a link with Green's function theory with two-particle and two-hole correlations. This means that we solve
\begin{equation}
0 = \langle ab \vert \hat{v} \vert ij \rangle + \left(\epsilon_a+\epsilon_b-\epsilon_i-\epsilon_j\right)t_{ij}^{ab}+\frac{1}{2}\sum_{cd} \langle ab \vert \hat{v} \vert cd \rangle t_{ij}^{cd}+\frac{1}{2}\sum_{kl} \langle kl \vert \hat{v} \vert ij \rangle t_{kl}^{ab}.
\label{eq:ccd2}
\end{equation}
This equation is solved the same way as we would do for Eq. (\ref{eq:ccd1}). The final step is then to include all terms in Eq. (\ref{eq:ccd}). 


\subsection{Formal derivation of the CCSD equations}


\section{Infinite nuclear matter}

\subsection{Introduction to studies of infinite matter}


Studies of infinite nuclear matter play an important role  in nuclear physics. The aim of this part of the lectures is to provide the necessary ingredients for perfoming studies of neutron star matter (or matter in $\beta$-equilibrium) and symmetric nuclear matter. We start however with the electron gas in two and three dimensions for both historical and pedagogical reasons. Since there are several benchmark calculations for the electron gas, this small detour will allow us to establish the necessary formalism. Thereafter we will study infinite nuclear matter 
\begin{itemize}
\item at the Hartree-Fock with realistic nuclear forces and

\item using many-body methods like coupled-cluster theory or in-medium SRG as discussed in our previous sections.
\end{itemize}

\noindent
\subsection{The infinite electron gas}

The electron gas is perhaps the only realistic model of a 
system of many interacting particles that allows for a solution
of the Hartree-Fock equations on a closed form. Furthermore, to first order in the interaction, one can also
compute on a closed form the total energy and several other properties of a many-particle systems. 
The model gives a very good approximation to the properties of valence electrons in metals.
The assumptions are

\begin{itemize}
 \item System of electrons that is not influenced by external forces except by an attraction provided by a uniform background of ions. These ions give rise to a uniform background charge. The ions are stationary.

 \item The system as a whole is neutral.

 \item We assume we have $N_e$ electrons in a cubic box of length $L$ and volume $\Omega=L^3$. This volume contains also a uniform distribution of positive charge with density $N_ee/\Omega$. 
\end{itemize}

\noindent
The homogeneous electron gas is one of the few examples of a system of many
interacting particles that allows for a solution of the mean-field
Hartree-Fock equations on a closed form.  To first order in the
electron-electron interaction, this applies to ground state properties
like the energy and its pertinent equation of state as well.  The
homogeneus electron gas is a system of electrons that is not
influenced by external forces except by an attraction provided by a
uniform background of ions. These ions give rise to a uniform
background charge.  The ions are stationary and the system as a whole
is neutral.
Irrespective of this simplicity, this system, in both two and
three-dimensions, has eluded a proper description of correlations in
terms of various first principle methods, except perhaps for quantum
Monte Carlo methods. In particular, the diffusion Monte Carlo
calculations of \href{{http://journals.aps.org/prl/abstract/10.1103/PhysRevLett.45.566}}{Ceperley} 
and \href{{http://journals.aps.org/prb/abstract/10.1103/PhysRevB.39.5005}}{Ceperley and Tanatar} 
are presently still considered as the
best possible benchmarks for the two- and three-dimensional electron
gas. 



The electron gas, in 
two or three dimensions is thus interesting as a test-bed for 
electron-electron correlations. The three-dimensional 
electron gas is particularly important as a cornerstone 
of the local-density approximation in density-functional 
theory. In the physical world, systems 
similar to the three-dimensional electron gas can be 
found in, for example, alkali metals and doped 
semiconductors. Two-dimensional electron fluids are 
observed on metal and liquid-helium surfaces, as well as 
at metal-oxide-semiconductor interfaces. However, the Coulomb 
interaction has an infinite range, and therefore 
long-range correlations play an essential role in the
electron gas. 




At low densities, the electrons become 
localized and form a lattice. This so-called Wigner 
crystallization is a direct consequence 
of the long-ranged repulsive interaction. At higher
densities, the electron gas is better described as a
liquid.
When using, for example, Monte Carlo methods the electron gas must be approximated 
by a finite system. The long-range Coulomb interaction 
in the electron gas causes additional finite-size effects  that are not
present in other infinite systems like nuclear matter or neutron star matter.
This poses additional challenges to many-body methods when applied 
to the electron gas.





\subsection{The infinite electron gas as a homogenous system}

This is a homogeneous system and the one-particle wave functions are given by plane wave functions normalized to a volume $\Omega$ 
for a box with length $L$ (the limit $L\rightarrow \infty$ is to be taken after we have computed various expectation values)
\[
\psi_{\mathbf{k}\sigma}(\mathbf{r})= \frac{1}{\sqrt{\Omega}}\exp{(i\mathbf{kr})}\xi_{\sigma}
\]
where $\mathbf{k}$ is the wave number and  $\xi_{\sigma}$ is a spin function for either spin up or down
\[ 
\xi_{\sigma=+1/2}=\left(\begin{array}{c} 1 \\ 0 \end{array}\right) \hspace{0.5cm}
\xi_{\sigma=-1/2}=\left(\begin{array}{c} 0 \\ 1 \end{array}\right).
\]




\subsection{Periodic boundary conditions}


We assume that we have periodic boundary conditions which limit the allowed wave numbers to
\[
k_i=\frac{2\pi n_i}{L}\hspace{0.5cm} i=x,y,z \hspace{0.5cm} n_i=0,\pm 1,\pm 2, \dots
\]
We assume first that the electrons interact via a central, symmetric and translationally invariant
interaction  $V(r_{12})$ with
$r_{12}=|\mathbf{r}_1-\mathbf{r}_2|$.  The interaction is spin independent.

The total Hamiltonian consists then of kinetic and potential energy
\[
\hat{H} = \hat{T}+\hat{V}.
\]
The operator for the kinetic energy can be written as
\[
\hat{T}=\sum_{\mathbf{k}\sigma}\frac{\hbar^2k^2}{2m}a_{\mathbf{k}\sigma}^{\dagger}a_{\mathbf{k}\sigma}.
\]



\subsection{Defining the Hamiltonian operator}

The Hamiltonian operator is given by
\[
\hat{H}=\hat{H}_{el}+\hat{H}_{b}+\hat{H}_{el-b},
\]
with the electronic part
\[
\hat{H}_{el}=\sum_{i=1}^N\frac{p_i^2}{2m}+\frac{e^2}{2}\sum_{i\ne j}\frac{e^{-\mu |\mathbf{r}_i-\mathbf{r}_j|}}{|\mathbf{r}_i-\mathbf{r}_j|},
\]
where we have introduced an explicit convergence factor
(the limit $\mu\rightarrow 0$ is performed after having calculated the various integrals).
Correspondingly, we have
\[
\hat{H}_{b}=\frac{e^2}{2}\int\int d\mathbf{r}d\mathbf{r}'\frac{n(\mathbf{r})n(\mathbf{r}')e^{-\mu |\mathbf{r}-\mathbf{r}'|}}{|\mathbf{r}-\mathbf{r}'|},
\]
which is the energy contribution from the positive background charge with density
$n(\mathbf{r})=N/\Omega$. Finally,
\[
\hat{H}_{el-b}=-\frac{e^2}{2}\sum_{i=1}^N\int d\mathbf{r}\frac{n(\mathbf{r})e^{-\mu |\mathbf{r}-\mathbf{x}_i|}}{|\mathbf{r}-\mathbf{x}_i|},
\]
is the interaction between the electrons and the positive background.



\subsection{Single-particle Hartree-Fock energy}

In the first exercise below we show that the Hartree-Fock energy can be written as 
\[
\varepsilon_{k}^{HF}=\frac{\hbar^{2}k^{2}}{2m_e}-\frac{e^{2}}
{\Omega^{2}}\sum_{k'\leq
k_{F}}\int d\mathbf{r}e^{i(\mathbf{k}'-\mathbf{k})\mathbf{r}}\int
d\mathbf{r'}\frac{e^{i(\mathbf{k}-\mathbf{k}')\mathbf{r}'}}
{\vert\mathbf{r}-\mathbf{r}'\vert}
\]
resulting in
\[
\varepsilon_{k}^{HF}=\frac{\hbar^{2}k^{2}}{2m_e}-\frac{e^{2}
k_{F}}{2\pi}
\left[
2+\frac{k_{F}^{2}-k^{2}}{kk_{F}}ln\left\vert\frac{k+k_{F}}
{k-k_{F}}\right\vert
\right]
\]



The previous result can be rewritten in terms of the density
\[
n= \frac{k_F^3}{3\pi^2}=\frac{3}{4\pi r_s^3},
\]
where $n=N_e/\Omega$, $N_e$ being the number of electrons, and $r_s$ is the radius of a sphere which represents the volum per conducting electron.  
It can be convenient to use the Bohr radius $a_0=\hbar^2/e^2m_e$.
For most metals we have a relation $r_s/a_0\sim 2-6$.  The quantity $r_s$ is dimensionless.


In the second exercise below  we find that
the total energy
$E_0/N_e=\langle\Phi_{0}|\hat{H}|\Phi_{0}\rangle/N_e$ for
for this system to first order in the interaction is given as 
\[
E_0/N_e=\frac{e^2}{2a_0}\left[\frac{2.21}{r_s^2}-\frac{0.916}{r_s}\right].
\]












\section{How to build a numerical quantum mechanical project}


\section{Summary}
 
