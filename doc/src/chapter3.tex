\title{Lattice quantum chromodynamics  approach to nuclear physics}\label{chap:latticeqcd}
\author{Tetsuo Hatsuda} 
\institute{Tetsuo Hatsuda \at Name of institution and address, \email{name@email.address}}
\maketitle
\abstract{Each chapter should be preceded by an abstract (10--15 lines long) that summarizes the content. The abstract will appear \textit{online} at \url{www.SpringerLink.com} and be available with unrestricted access. This allows unregistered users to read the abstract as a teaser for the complete chapter. As a general rule the abstracts will not appear in the printed version of your book unless it is the style of your particular book or that of the series to which your book belongs.\newline\indent
Please use the 'starred' version of the new Springer \texttt{abstract} command for typesetting the text of the online abstracts (cf. source file of this chapter template \texttt{abstract}) and include them with the source files of your manuscript. Use the plain \texttt{abstract} command if the abstract is also to appear in the printed version of the book.}

\section{General Introduction}
\section{Continuum quantum chromodynamics: basics} 
\section{Lattice quantum chromodynamics: basics}
\section{Lattice quantum chromodynamics: applications}
\section{Hadron interactions: basics}


\begin{acknowledgement}
If you want to include acknowledgments of assistance and the like at the end of an individual chapter please use the \verb|acknowledgement| environment -- it will automatically render Springer's preferred layout.
\end{acknowledgement}
%
\section*{Appendix}
\addcontentsline{toc}{section}{Appendix}

\begin{thebibliography}{99.}%
\bibitem{ts1} First reference
\end{thebibliography}
