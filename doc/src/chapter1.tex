\title{Motivation and overarching aims}
\author{Morten Hjorth-Jensen, Maria Paola Lombardo, and Ubirajara van Kolck}
\institute{
Morten Hjorth-Jensen  \at Department of Physics and Astronomy and National Superconducting Cyclotron Laboratory, Michigan State University, East Lansing, Michigan, USA and Department of Physics, University of Oslo, Oslo, Norway, \email{hjensen@msu.edu}, \and
%
Maria Paola Lombardo \at INFN, Laboratori Nazionali di Frascati, Frascati, Italy, \email{mariapaola.lombardo@lnf.infn.it}, \and
%
Ubirajara van Kolck \at Institut de Physique Nucleaire, Orsay, France and Department of Physics, University of Arizona, Tucson, Arizona, USA, \email{vankolck@ipno.in2p3.fr}
}

\maketitle


Nuclear physics has recently experienced several discoveries and
technological advances that address the fundamental questions of the
field, in particular how nuclei emerge from the strong dynamics
of quantum chromodynamics (QCD).
Many of these advances have been made possible by significant
investments in frontier research facilities worldwide over the last
two decades. Some of these discoveries are the detection of perhaps
the most exotic state of matter, the quark-gluon plasma, which is
believed to have existed in the very first moments of the Universe.  
Recent experiments have validated the standard solar model
and established that neutrinos have mass. High-precision
measurements of the quark structure of the nucleon are challenging
existing theoretical understanding.  Nuclear physicists have started
to explore a completely unknown landscape of nuclei with extreme
neutron-to-proton ratios using radioactive and short-lived ions,
including rare and very neutron-rich isotopes.  These experiments push
us towards the extremes of nuclear stability.  Moreover, these rare
nuclei lie at the heart of nucleosynthesis processes in the Universe
and are therefore an important component in the puzzle of matter
generation in the Universe.

A firm experimental and theoretical understanding of nuclear stability
in terms of the basic constituents is a huge intellectual endeavor.
Experiments indicate that developing a comprehensive description of
all nuclei and their reactions requires theoretical and experimental
investigations of rare isotopes with unusual neutron-to-proton ratios
that are very different from their stable counterparts.  These rare
nuclei are difficult to produce and study experimentally since they
can have extremely short lifetimes. 
To study theoretically these nuclear systems  entails 
being able to solve a complicated quantum-mechanical many-body problem
in order to address important 
issues such as whether we can explain from first-principle
methods the existence of magic numbers and their eventual vanishing with increasing neutron numbers, 
how the binding energy of neutron-rich nuclei
behaves, or the radii, neutron skins, and many many other probes that
extract information about many-body correlations as nuclei evolve
towards their limits of stability. These are all fundamental
questions which, combined with recent experimental and theoretical
advances, will allow us to advance our basic knowledge about the
limits of stability of matter, and, hopefully, help us in gaining a
better understanding of visible matter.


Accompanying the experimental  developments, a qualitative change has swept the
nuclear theory landscape thanks to a combination of techniques that are
allowing, for the first time, to construct links between QCD and
the nuclear many-body problem. This transformation has been brought by a dramatic
improvement in the capability of numerical calculations both in QCD,
via lattice simulations, and in the nuclear many-body problem via first principle or {\em ab initio} 
many-body methods that employ non-relativistic
Hamiltonians. Simultaneously, effective field
theories attempt at building  a bridge between the two numerical approaches,
allowing to convert the results of lattice QCD into input Hamiltonians that can be used in {\em ab initio}
methods.
Furthermore,
algorithmic and computational advances hold promise for
breakthroughs in predictive power including proper error estimates,
enhancing the already strong ties between theory and experiment.
These advances include better {\em ab initio} many-body methods as well as a
better understanding of the underlying effective degrees of freedom
and the respective forces at play.  Similarly, we have recently witnessed a significant improvement in numerical
algorithms and high-performance computing.
This provides us with important new insights about the stability
of nuclear matter and allows us to relate these novel understandings to
the underlying laws of motion, the corresponding forces and the
pertinent fundamental building blocks of nuclear matter.


It is within this framework the present set of lectures finds its rationale.
This text collects and synthesizes ten series of lectures on the
nuclear many-body problem, starting from our present understanding of
the underlying forces with a presentation of recent advances within
the field of lattice QCD, via effective field theories to central
many-body methods like various Monte Carlo approaches, coupled-cluster
theory, the similarity renormalization group approach, Green's
function methods and large-scale diagonalization methods.  The
applications span from our smallest components, quarks and gluons as
the mediators of the strong force to the computation of the equation
of state for infinite nuclear matter and neutron star matter.  The
lectures provide a proper exposition of the underlying theoretical and
algorithmic approaches as well as strong ties to the numerical
implementation of the exposed methods.  

The next chapter, by Thomas Sch\"afer, aims at a brief introduction to
quantum chromodynamics (QCD), the QCD phase diagram, and
non-equilibrium phenomena in QCD. This chapter emphasizes the aspects
of the theory that can be addressed using computational methods. In
chapter 3, Tetsuo Hatsuda presents several basic concepts and
applications of lattice quantum chromodynamics (LQCD), ending the
chapter by presenting recent LQCD results on baryon-baryon
interactions.  These results are extremely promising since they allow
for a better understanding of the links between QCD and effective
field theories with say nucleons and pions only. The latter provide
the necessary degrees of freedom and inputs for defining the nuclear
Hamiltonians that enter the solution of the various nuclear many-body
methods discussed in chapters 7 through 11. Chapter 4 by Hans-Werner
Hammer and Sebastian K\"onig presents the general theoretical aspects
of nuclear effective field theories. In the two subsequent chapters by
Amy Nicholson and Dean Lee, the authors apply lattice techniques to
nuclear effective field theories involving nucleons and pions as the
basic degrees of freedom. These authors give a detailed exposure, with
exercises and numerical codes, of lattice techniques applied to
effective field theory, explaining the theory and algorithms relevant
to lattice simulations of nuclear few- and many-body systems.  Chapter
7 by Giuseppina Orlandini gives an overview of several {\it ab initio}
approaches currently used to study nuclear structure properties and
reactions. Chapter 8 by Justin Lietz, Sam Novario {\em et al}
introduces a computational approach to infinite nuclear matter
employing Hartree-Fock theory, many-body perturbation theory and
coupled cluster theory, with an extensive discussion of computational
topics. Many of these basic ingredients are used in the next three
chapters. Chapter 9 by Francesco Pederiva, Alessandro Roggero and
Kevin Schmidt reviews Quantum Monte Carlo methods for solving the
many-body Schr\"odinger equation for an arbitrary Hamiltonian ending
the discussion with the newly developed Configuration Interaction
Monte Carlo algorithm. Comparisons are made  with coupled cluster theory for
the equation of state of infinite neutron star matter from chapter 8. Chapter 10 by
Heiko Hergert {\em et al} presents applications of the In-Medium
Similarity Renormalization Group method to studies of infinite nuclear
matter. Finally chapter 11, by Carlo Barbieri and Arianna Carbone,
presents the fundamental techniques and working equations of many-body
Green's function theory for calculating ground state properties and
the spectral strength, with applications to infinite neutron star
matter and comparisons with the results from chapters 8-10.


The first five chapters are thus meant to expose the reader to the
most recent developments in our understanding of the strong
interaction, linking QCD with effective field theories. With the
appropriate and pertinent effective degrees of freedom we can in turn define various
effective non-relativistic Hamiltonians and embark on our studies of
widely used methods for solving the non-relativistic Schr\"odinger
equation. Spanning from Monte Carlo methods to various wave function
based methods like full configuration interaction theory, coupled
cluster theory, similarity renormalization group approaches and
Green's function theory, chapters 7 through 11 aim at presenting these
methods to the reader, with applications to infinite nuclear matter in
chapters 8 through 11. Studies of infinite matter play a central role
in nuclear physics. The determination of for example the equation of
state (EoS), which is intimately linked with our capability to solve
the nuclear many-body problem, has important consequences for neutron
star properties like the mass range, the mass-radius relationship, the
thickness of the crust and the rate by which a neutron star cools down
over time. The EoS is also an important ingredient in studies of the
energy release in supernova explosions. Infinite matter offers also
several technical simplifications to the many-body problem compared
with finite nuclei, as discussed in chapters 8 through 11. In these chapters we
provide benchmark calculations and compare different many-body methods
using a simplified model for the nuclear forces. However, the
formalism and codes we present can easily be extended to include
interaction models based on effective field theories, as well as other
systems, spanning from the homogeneous electron gas in two and three
dimensions to finite systems like nuclei.  The various chapters
propose exercises meant to deepen the  theoretical concepts that are discussed.
Actual numerical software allows the reader to build upon the
theoretical concepts and develop her/his own insights about these
methods. These codes can serve as a starting point for developing
own programs for tackling complicated many-body problems.
Proper benchmarks for the various programs are also provided, allowing
thereby potential readers and users to check the correctness,
installation and compilation of the various programs. All codes are
properly linked and available via the github
link \url{https://github.com/ManyBodyPhysics/LectureNotesPhysics/tree/master/Programs}.


\begin{acknowledgement}
The different chapters are based on lectures given at the Doctoral Training program {\em Computational Nuclear Physics - Hadrons, Nuclei and Dense Matter} held at  the European Center for Theoretical Nuclear Physics
and Related Areas (ECT*) in Trento, Italy, from April 13 to May 22, 2015, the Nuclear Talent course {\em Many-body methods for nuclear physics}, held at Grand Accelerateur National d’Ions Lourds (GANIL), Caen, France, from July 5 to July 25,  2015, and the Nuclear Talent course {\em High-performance computing and computational tools for nuclear physics} held at North Carolina State University from July 11 to July 29, 2016. For more information about the Nuclear Talent courses see \url{http://www.nucleartalent.org}. For the Doctoral Training program of the ECT*, see \url{http://www.ectstar.eu/node/799}.
The support for organizing these series of lectures from the ECT*, GANIL and the University of Basse Normadie at Caen, North Carolina State University, Los Alamos National Laboratory and Michigan State University is greatly acknowledged.   Two of the lectures (chapters 8 and 10) are co-authored by graduate students (from Michigan State University) who  attended the abovementioned Nuclear Talent courses.


The work of MHJ is supported by NSF Grant No.~PHY-1404159 (Michigan State University).

The work of UvK is supported in part by the European Union Research
and Innovation program Horizon 2020 under grant number 654002 and by
the DOE, Office of Science, Office of Nuclear Physics, under award
number DE-FG02-04ER41338.

\end{acknowledgement}




