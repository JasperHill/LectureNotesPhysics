\preface
This graduate-level text collects and synthesizes ten series of
lectures on the nuclear quantum many-body problem - starting from our
present understanding of the underlying forces with a presentation of
recent advances within the field of lattice quantum chromodynamics,
via effective field theories to central many-body methods like Monte
Carlo methods, coupled cluster theories, the similarity renormalization group approach, Green's function methods 
and large-scale
diagonalization approaches.

In particular algorithmic and computational advances show promise for
breakthroughs in predictive power including proper error estimates, a
better understanding of the underlying effective degrees of freedom
and of the respective forces at play.

Enabled by recent advances in theoretical, experimental and numerical
techniques, the modern and state-of-the art applications considered in
this volume span the entire range from our smallest components, quarks
and gluons as the mediators of the strong force to the computation of
the equation of state for neutron star matter.

 

The present lectures provide a proper exposition of the underlying
theoretical and algorithmic approaches as well as strong ties to the
numerical implementation of the exposed methods. Several of the
lectures provide links to actual numerical software and benchmark
calculations, allowing eventual readers, based upon the available
material, to develop their own programs for tackling challenging
nuclear many-body problems.

